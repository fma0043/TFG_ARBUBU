\documentclass[a4paper,12pt,twoside]{memoir}

% Castellano
\usepackage[spanish,es-tabla]{babel}
\selectlanguage{spanish}
\usepackage[utf8]{inputenc}
\usepackage[T1]{fontenc}
\usepackage{lmodern} % scalable font
\usepackage{microtype}
\usepackage{placeins}
\usepackage{longtable,booktabs}

\RequirePackage{booktabs}
\RequirePackage[table]{xcolor}
\RequirePackage{xtab}
\RequirePackage{multirow}

% Multi-page tables using
\usepackage{longtable}

% Muestra un directorio en el árbol
\newcommand{\desc}[1]{
	\ldots{} \begin{minipage}[t]{8cm}
		#1{.}
	\end{minipage}
}

% Links
\usepackage[colorlinks]{hyperref}
\hypersetup{
	allcolors = {red}
}

% Ecuaciones
\usepackage{amsmath}

% Rutas de fichero / paquete
\newcommand{\ruta}[1]{{\sffamily #1}}

% Párrafos
\nonzeroparskip


% Imagenes
\usepackage{graphicx}
\newcommand{\imagen}[2]{
	\begin{figure}[!h]
		\centering
		\includegraphics[width=0.9\textwidth]{#1}
		\caption{#2}\label{fig:#1}
	\end{figure}
	\FloatBarrier
}

\newcommand{\imagenflotante}[2]{
	\begin{figure}%[!h]
		\centering
		\includegraphics[width=0.9\textwidth]{#1}
		\caption{#2}\label{fig:#1}
	\end{figure}
}

\newcommand{\imagenPequena}[2]{
	\begin{figure}[!h]
		\centering
		\includegraphics[width=0.3\textwidth]{#1}
	\end{figure}
	\FloatBarrier
}

\newcommand{\insertarimagen}[3]{
	\begin{figure}[!h]
		\centering
		\includegraphics[width=0.9\textwidth]{#1}
		\caption[#2]{#2 \cite{#3}}\label{fig:#1}
	\end{figure}
	\FloatBarrier
}
% El comando \figura nos permite insertar figuras comodamente, y utilizando
% siempre el mismo formato. Los parametros son:
% 1 -> Porcentaje del ancho de página que ocupará la figura (de 0 a 1)
% 2 --> Fichero de la imagen
% 3 --> Texto a pie de imagen
% 4 --> Etiqueta (label) para referencias
% 5 --> Opciones que queramos pasarle al \includegraphics
% 6 --> Opciones de posicionamiento a pasarle a \begin{figure}
\newcommand{\figuraConPosicion}[6]{%
	\setlength{\anchoFloat}{#1\textwidth}%
	\addtolength{\anchoFloat}{-4\fboxsep}%
	\setlength{\anchoFigura}{\anchoFloat}%
	\begin{figure}[#6]
		\begin{center}%
			\Ovalbox{%
				\begin{minipage}{\anchoFloat}%
					\begin{center}%
						\includegraphics[width=\anchoFigura,#5]{#2}%
						\caption{#3}%
						\label{#4}%
					\end{center}%
				\end{minipage}
			}%
		\end{center}%
	\end{figure}%
}

%
% Comando para incluir imágenes en formato apaisado (sin marco).
\newcommand{\figuraApaisadaSinMarco}[5]{%
	\begin{figure}%
		\begin{center}%
			\includegraphics[angle=90,height=#1\textheight,#5]{#2}%
			\caption{#3}%
			\label{#4}%
		\end{center}%
	\end{figure}%
}
% Para las tablas
\newcommand{\otoprule}{\midrule [\heavyrulewidth]}
%
% Nuevo comando para tablas pequeñas (menos de una página).
\newcommand{\tablaSmall}[5]{%
	\begin{table}
		\begin{center}
			\rowcolors {2}{gray!35}{}
			\begin{tabular}{#2}
				\toprule
				#4
				\otoprule
				#5
				\bottomrule
			\end{tabular}
			\caption{#1}
			\label{tabla:#3}
		\end{center}
	\end{table}
}

%
%Para el float H de tablaSmallSinColores
\usepackage{float}

%
% Nuevo comando para tablas pequeñas (menos de una página).
\newcommand{\tablaSmallSinColores}[5]{%
	\begin{table}[H]
		\begin{center}
			\begin{tabular}{#2}
				\toprule
				#4
				\otoprule
				#5
				\bottomrule
			\end{tabular}
			\caption{#1}
			\label{tabla:#3}
		\end{center}
	\end{table}
}

\newcommand{\tablaApaisadaSmall}[5]{%
	\begin{landscape}
		\begin{table}
			\begin{center}
				\rowcolors {2}{gray!35}{}
				\begin{tabular}{#2}
					\toprule
					#4
					\otoprule
					#5
					\bottomrule
				\end{tabular}
				\caption{#1}
				\label{tabla:#3}
			\end{center}
		\end{table}
	\end{landscape}
}

%
% Nuevo comando para tablas grandes con cabecera y filas alternas coloreadas en gris.
\newcommand{\tabla}[6]{%
	\begin{center}
		\tablefirsthead{
			\toprule
			#5
			\otoprule
		}
		\tablehead{
			\multicolumn{#3}{l}{\small\sl continúa desde la página anterior}\\
			\toprule
			#5
			\otoprule
		}
		\tabletail{
			\hline
			\multicolumn{#3}{r}{\small\sl continúa en la página siguiente}\\
		}
		\tablelasttail{
			\hline
		}
		\bottomcaption{#1}
		\rowcolors {2}{gray!35}{}
		\begin{xtabular}{#2}
			#6
			\bottomrule
		\end{xtabular}
		\label{tabla:#4}
	\end{center}
}

%
% Nuevo comando para tablas grandes con cabecera.
\newcommand{\tablaSinColores}[6]{%
	\begin{center}
		\tablefirsthead{
			\toprule
			#5
			\otoprule
		}
		\tablehead{
			\multicolumn{#3}{l}{\small\sl continúa desde la página anterior}\\
			\toprule
			#5
			\otoprule
		}
		\tabletail{
			\hline
			\multicolumn{#3}{r}{\small\sl continúa en la página siguiente}\\
		}
		\tablelasttail{
			\hline
		}
		\bottomcaption{#1}
		\begin{xtabular}{#2}
			#6
			\bottomrule
		\end{xtabular}
		\label{tabla:#4}
	\end{center}
}

%
% Nuevo comando para tablas grandes sin cabecera.
\newcommand{\tablaSinCabecera}[5]{%
	\begin{center}
		\tablefirsthead{
			\toprule
		}
		\tablehead{
			\multicolumn{#3}{l}{\small\sl continúa desde la página anterior}\\
			\hline
		}
		\tabletail{
			\hline
			\multicolumn{#3}{r}{\small\sl continúa en la página siguiente}\\
		}
		\tablelasttail{
			\hline
		}
		\bottomcaption{#1}
		\begin{xtabular}{#2}
			#5
			\bottomrule
		\end{xtabular}
		\label{tabla:#4}
	\end{center}
}



\definecolor{cgoLight}{HTML}{EEEEEE}
\definecolor{cgoExtralight}{HTML}{FFFFFF}

%
% Nuevo comando para tablas grandes sin cabecera.
\newcommand{\tablaSinCabeceraConBandas}[5]{%
	\begin{center}
		\tablefirsthead{
			\toprule
		}
		\tablehead{
			\multicolumn{#3}{l}{\small\sl continúa desde la página anterior}\\
			\hline
		}
		\tabletail{
			\hline
			\multicolumn{#3}{r}{\small\sl continúa en la página siguiente}\\
		}
		\tablelasttail{
			\hline
		}
		\bottomcaption{#1}
		\rowcolors[]{1}{cgoExtralight}{cgoLight}
		
		\begin{xtabular}{#2}
			#5
			\bottomrule
		\end{xtabular}
		\label{tabla:#4}
	\end{center}
}



\newcommand{\TablaCasoDeUso}[9]{
	\begin{longtable}[H]{@{}ll@{}}
		\toprule
		\begin{minipage}[b]{0.25\columnwidth}\raggedright\strut
			\textbf{CU-#1}\strut
		\end{minipage} & \begin{minipage}[b]{0.70\columnwidth}\raggedright\strut
			\textbf{#2}\strut
		\end{minipage}\tabularnewline
		\midrule
		\endhead
		\begin{minipage}[t]{0.25\columnwidth}\raggedright\strut
			\textbf{Versión}\strut
		\end{minipage} & \begin{minipage}[t]{0.70\columnwidth}\raggedright\strut
			1.0\strut
		\end{minipage}\tabularnewline
		\begin{minipage}[t]{0.25\columnwidth}\raggedright\strut
			\textbf{Autor}\strut
		\end{minipage} & \begin{minipage}[t]{0.70\columnwidth}\raggedright\strut
			\nombre\strut
		\end{minipage}\tabularnewline
		\begin{minipage}[t]{0.25\columnwidth}\raggedright\strut
			\textbf{Requisitos asociados}\strut
		\end{minipage} & \begin{minipage}[t]{0.70\columnwidth}\raggedright\strut
			#3\strut
		\end{minipage}\tabularnewline
		\begin{minipage}[t]{0.25\columnwidth}\raggedright\strut
			\textbf{Descripción}\strut
		\end{minipage} & \begin{minipage}[t]{0.70\columnwidth}\raggedright\strut
			#4\strut
		\end{minipage}\tabularnewline
		\begin{minipage}[t]{0.25\columnwidth}\raggedright\strut
			\textbf{Precondiciones}\strut
		\end{minipage} & \begin{minipage}[t]{0.70\columnwidth}\raggedright\strut
			#5\strut
		\end{minipage}\tabularnewline
		\begin{minipage}[t]{0.25\columnwidth}\raggedright\strut
			\textbf{Acciones}\strut
		\end{minipage} & \begin{minipage}[t]{0.70\columnwidth}\raggedright\strut
			\begin{enumerate}
				\def\labelenumi{\arabic{enumi}.}
				\tightlist
				#6
			\end{enumerate}\strut
		\end{minipage}\tabularnewline
		\begin{minipage}[t]{0.25\columnwidth}\raggedright\strut
			\textbf{Postcondiciones}\strut
		\end{minipage} & \begin{minipage}[t]{0.70\columnwidth}\raggedright\strut
			#7\strut
		\end{minipage}\tabularnewline
		\begin{minipage}[t]{0.25\columnwidth}\raggedright\strut
			\textbf{Excepciones}\strut
		\end{minipage} & \begin{minipage}[t]{0.70\columnwidth}\raggedright\strut
			\begin{itemize}
				\tightlist
				#8
			\end{itemize}\strut
		\end{minipage}\tabularnewline
		\begin{minipage}[t]{0.25\columnwidth}\raggedright\strut
			\textbf{Importancia}\strut
		\end{minipage} & \begin{minipage}[t]{0.70\columnwidth}\raggedright\strut
			#9\strut
		\end{minipage}\tabularnewline
		\bottomrule
		\caption{CU-#1 #2.}
		\label{CU:#1}
	\end{longtable}
	\newpage
}



\graphicspath{ {./img/} }

% Capítulos
\chapterstyle{bianchi}
\newcommand{\capitulo}[2]{
	\setcounter{chapter}{#1}
	\setcounter{section}{0}
	\chapter*{#2}
	\addcontentsline{toc}{chapter}{#2}
	\markboth{#2}{#2}
}

% Apéndices
\renewcommand{\appendixname}{Apéndice}
\renewcommand*\cftappendixname{\appendixname}

\newcommand{\apendice}[1]{
	%\renewcommand{\thechapter}{A}
	\chapter{#1}
}

\renewcommand*\cftappendixname{\appendixname\ }

% Formato de portada
\makeatletter
\usepackage{xcolor}
\newcommand{\tutor}[1]{\def\@tutor{#1}}
\newcommand{\course}[1]{\def\@course{#1}}
\definecolor{cpardoBox}{HTML}{E6E6FF}
\def\maketitle{
	\null
	\thispagestyle{empty}
	% Cabecera ----------------
	\noindent\includegraphics[width=\textwidth]{cabecera}\vspace{1cm}%
	\vfill
	% Título proyecto y escudo informática ----------------
	\colorbox{cpardoBox}{%
		\begin{minipage}{.8\textwidth}
			\vspace{.5cm}\Large
			\begin{center}
				\textbf{TFG del Grado en Ingeniería Informática}\vspace{.6cm}\\
				\textbf{\LARGE\@title{}}
			\end{center}
			\vspace{.2cm}
		\end{minipage}
		
	}%
	\hfill\begin{minipage}{.20\textwidth}
		\includegraphics[width=\textwidth]{escudoInfor}
	\end{minipage}
	\vfill
	% Datos de alumno, curso y tutores ------------------
	\begin{center}%
		{%
			\imagenPequena{portada/logo}
			\noindent\LARGE
			Presentado por \@author{}\\ 
			en Universidad de Burgos --- \@date{}\\
			Tutor: \@tutor{}\\
		}%
	\end{center}%
	\null
	\cleardoublepage
}
\makeatother

\newcommand{\nombre}{Félix Movilla Alonso} %%% cambio de comando
\newcommand{\nombreTutor}{Pedro Renedo Fernández y Antonio Jesús Canepa Oneto}

% Datos de portada
\title{ARBUBU}
\author{\nombre}
\tutor{\nombreTutor}
\date{\today}

\begin{document}
	
	\maketitle
	
	
	
	\cleardoublepage
	
	
	
	%%%%%%%%%%%%%%%%%%%%%%%%%%%%%%%%%%%%%%%%%%%%%%%%%%%%%%%%%%%%%%%%%%%%%%%%%%%%%%%%%%%%%%%%
	
	
	
	\frontmatter
	
	
	\clearpage
	
	% Indices
	\tableofcontents
	
	\clearpage
	
	\listoffigures
	
	\clearpage
	
	\listoftables
	
	\clearpage
	
	\mainmatter
	
	\appendix
	
	\apendice{Plan de Proyecto Software}

\section{Introducción}

Todo proyecto considerado importante tiene que tener una buena planificación, en dicha planificación fijaremos los requisitos, el tiempo estimado y el dinero que creemos que va a costar realizar dicho proyecto.

El plan de proyecto software estará contenido por una planificación temporal y un estudio de viabilidad que nos dirá si el proyecto saldrá rentable o no.

\section{Planificación temporal}

Como decidí seguir una metodología en espiral \cite{ModeloenEspiral} el proyecto se divide en ciclos que a su vez se divide en 4 fases. 

\imagen{planProyecto/modeloEnEspiral}{Metodología en Espiral}

\subsection{Ciclo 1}

	\begin{enumerate}
		\item \textbf{Planificación}: Se decide en consenso con los tutores las herramientas a utilizar.
		\item \textbf{Análisis}: Al elegir una herramienta u otra se toman una serie de ventajas e inconvenientes y se decide optar por las que más me llaman la atención y las que considero como se explica en la memoria mas adecuadas para el proyecto.
		\item \textbf{Implementación}: Se instalan dichas herramientas y se empieza a trastear con ellas y se visualizan manuales para el correcto aprendizaje de su funcionamiento.
		\item \textbf{Evaluación}: Antes de pasar a otro ciclo se revisa las otras tres fases para ver si todo ha ido correctamente, se cree que las herramientas utilizadas tras manejarlas son las correctas y se cambia de ciclo.
	\end{enumerate}


\subsection{Ciclo 2}

\begin{enumerate}
	\item \textbf{Planificación}: Se tiene una reunión con los tutores para ver que es lo siguiente a realizar y se decide empezar por crear un proyecto y añadir documentación en Latex\cite{Latex}.
	\item \textbf{Análisis}: Se evalúa que lo que se va a realizar en este ciclo es lo correcto y se decide seguir con su desarrollo.
	\item \textbf{Implementación}: Se crea el proyecto en Django\cite{Django}, se crea los primeros modelos, se realiza el resumen y se añaden los objetivos del proyecto.
	\item \textbf{Evaluación}: Antes de pasar a otro ciclo se revisa las otras tres fases para ver si todo ha ido correctamente, se cree que el trabajo realizado es el correcto y se cambia de ciclo.
\end{enumerate}

\subsection{Ciclo 3}

\begin{enumerate}
	\item \textbf{Planificación}: Se tiene una reunión con los tutores para ver que es lo siguiente a realizar y se decide modificar los modelos.
	\item \textbf{Análisis}: Se evalúa que lo que se va a realizar en este ciclo es lo correcto y se decide seguir con su desarrollo.
	\item \textbf{Implementación}: Se modifica los modelos realizados anteriormente ante las exigencias del cliente.
	\item \textbf{Evaluación}: Antes de pasar a otro ciclo se revisa las otras tres fases para ver si todo ha ido correctamente, se cree que el trabajo realizado es el correcto y se cambia de ciclo.
\end{enumerate}

\subsection{Ciclo 4}

\begin{enumerate}
	\item \textbf{Planificación}: Se tiene una reunión con los tutores para ver que es lo siguiente a realizar y se decide crear unos métodos para ser utilizados más adelante.
	\item \textbf{Análisis}: Se evalúa que lo que se va a realizar en este ciclo es lo correcto y se decide seguir con su desarrollo.
	\item \textbf{Implementación}: Se crean unos métodos, pero cuando es enseñado al tutor se considera que no están bien realizados y se borran y se decide pasar a otro ciclo y dejar esto para más adelante.
	\item \textbf{Evaluación}: Antes de pasar a otro ciclo se revisa las otras tres fases y al considerar que la creación de los métodos no eran correctos se decide dejar para más adelante y se cambia de ciclo.
\end{enumerate}

\subsection{Ciclo 5}

\begin{enumerate}
	\item \textbf{Planificación}: Se tiene una reunión con los tutores para ver que es lo siguiente a realizar y se decide añadir el mapa y los primeros árboles.
	\item \textbf{Análisis}: Se evalúa que lo que se va a realizar en este ciclo es lo correcto y se decide seguir con su desarrollo.
	\item \textbf{Implementación}: Se añade el mapa y los primeros árboles.
	\item \textbf{Evaluación}: Antes de pasar a otro ciclo se revisa las otras tres fases para ver si todo ha ido correctamente, se cree que el trabajo realizado es el correcto y se cambia de ciclo.
\end{enumerate}

\subsection{Ciclo 6}

\begin{enumerate}
	\item \textbf{Planificación}: Se tiene una reunión con los tutores para ver que es lo siguiente a realizar y se decide avanzar con el diseño de la página y seguir avanzando con la documentación.
	\item \textbf{Análisis}: Se evalúa que lo que se va a realizar en este ciclo es lo correcto y se decide seguir con su desarrollo.
	\item \textbf{Implementación}: Se sigue avanzando en el diseño de la página y se crean los apartados de requisitos, técnicas metodológicas, conceptos teóricos, técnicas y herramientas y trabajos relacionados de la documentación.
	\item \textbf{Evaluación}: Antes de pasar a otro ciclo se revisa las otras tres fases para ver si todo ha ido correctamente, se cree que el trabajo realizado es el correcto y se cambia de ciclo.
\end{enumerate}

\subsection{Ciclo 7}

\begin{enumerate}
	\item \textbf{Planificación}: Se tiene una reunión con los tutores para ver que es lo siguiente a realizar y se decide crear la gestión de los usuarios.
	\item \textbf{Análisis}: Se evalúa que lo que se va a realizar en este ciclo es lo correcto y se decide seguir con su desarrollo.
	\item \textbf{Implementación}: Se crea la gestión de usuarios para poder iniciar sesión, registrarse y cerrar sesión.
	\item \textbf{Evaluación}: Antes de pasar a otro ciclo se revisa las otras tres fases para ver si todo ha ido correctamente, se cree que el trabajo realizado es el correcto y se cambia de ciclo.
\end{enumerate}

\subsection{Ciclo 8}

\begin{enumerate}
	\item \textbf{Planificación}: Se tiene una reunión con los tutores para ver que es lo siguiente a realizar y se decide ir introduciendo las fotos de los modelos en la base de datos, además de diseñar las pantallas de las especies, individuos, géneros y familias.
	\item \textbf{Análisis}: Se evalúa que lo que se va a realizar en este ciclo es lo correcto y se decide seguir con su desarrollo.
	\item \textbf{Implementación}: Se introduce las fotos de los modelos en la base de datos y se diseñan las pantallas de especies, individuos, géneros y familias.
	\item \textbf{Evaluación}: Antes de pasar a otro ciclo se revisa las otras tres fases para ver si todo ha ido correctamente, se cree que el trabajo realizado es el correcto y se cambia de ciclo.
\end{enumerate}

\subsection{Ciclo 9}

\begin{enumerate}
	\item \textbf{Planificación}: Se tiene una reunión con los tutores para ver que es lo siguiente a realizar y se decide ir introduciendo los árboles en los distintos mapas de la aplicación a través de ficheros javascript.
	\item \textbf{Análisis}: Se evalúa que lo que se va a realizar en este ciclo es lo correcto y se decide seguir con su desarrollo.
	\item \textbf{Implementación}: Se introducen los árboles en los distintos mapas mediante llamadas a ficheros javascript.
	\item \textbf{Evaluación}: Antes de pasar a otro ciclo se revisa las otras tres fases para ver si todo ha ido correctamente, se cree que el trabajo realizado es el correcto y se cambia de ciclo.
\end{enumerate}

\subsection{Ciclo 10}

\begin{enumerate}
	\item \textbf{Planificación}: Se tiene una reunión con los tutores para ver que es lo siguiente a realizar y se decide agregar una opción para añadir individuos mediante formularios y exportarlos en pdf.
	\item \textbf{Análisis}: Se evalúa que lo que se va a realizar en este ciclo es lo correcto y se decide seguir con su desarrollo.
	\item \textbf{Implementación}: Se introduce una opción para que un usuario registrado pueda añadir individuos y también para que pueda descargar un pdf con los árboles y sus características y localización.
	\item \textbf{Evaluación}: Antes de pasar a otro ciclo se revisa las otras tres fases para ver si todo ha ido correctamente, se cree que el trabajo realizado es el correcto y se cambia de ciclo.
\end{enumerate}

\subsection{Ciclo 11}

\begin{enumerate}
	\item \textbf{Planificación}: Se tiene una reunión con los tutores para ver que es lo siguiente a realizar y una vez terminada el diseño de las pantallas de aplicación se decide seguir con la documentación de los anexos.
	\item \textbf{Análisis}: Se evalúa que lo que se va a realizar en este ciclo es lo correcto y se decide seguir con su desarrollo.
	\item \textbf{Implementación}: Se continúa con la documentación de los anexos.
	\item \textbf{Evaluación}: Antes de pasar a otro ciclo se revisa las otras tres fases para ver si todo ha ido correctamente, se cree que el trabajo realizado es el correcto y se cambia de ciclo.
\end{enumerate}

\section{Estudio de viabilidad}

En este apartado se va a analizar la viabilidad económica y la viabilidad legal.


\subsection{Viabilidad económica}

Si se implantase el desarrollo en un entorno empresarial real los costes serían los siguientes:

\begin{itemize}
	\item \textbf{Costes de personal}: Teniendo en cuenta que el proyecto ha tenido una duración aproximada de 6 meses, considerando que lo lleva a cabo un desarrollador con un salario bruto de 1300\textup{\euro} mensuales contratado a tiempo parcial, con unas contingencias comunes\cite{seguridadSocial} de 28.3\%  y una retención del IRPF de 10.49\%, el salario neto será de 795.73\textup{\euro} y el coste total será de 7800\textup{\euro}.
	
	\item \textbf{Costes de material}: Hay que tener en cuenta el coste de hardware y coste de software, el coste del software es gratuito ya que como hemos mencionado en la memoria todas las herramientas utilizadas han sido de software libre y gratuito, el coste del hardware es la utilización de un portátil, contando que el coste del ordenador fue de 600\textup{\euro} y con una antigüedad de aproximadamente 4 años y el tiempo utilizado de 6 meses, el coste amortizado sería de 60\textup{\euro}.
	
	\item \textbf{Costes de material}: La suma del coste de personal y los costes de material es de 7860\textup{\euro}.

\end{itemize}

	Teniendo en cuenta los resultados obtenidos podemos concluir que el proyecto resultará rentable a partir de unos dos años de vida.
	
\subsection{Viabilidad legal}

	En este apartado se expondrá la viabilidad legal del proyecto, al utilizar Python que posee licencia PSFL\cite{licencia}, es de software libre y cumple los requisitos OSI y es compatible con licencia GPL.
	
	Además todo el software utilizado ha sido de software libre, así que la licencia que se adapta mejor al proyecto es la GNU\cite{gnu}, esta licencia permite el uso, distribución y modificación siempre y cuando no se modifique la dicha licencia y acredite al autor.
	
	
	\apendice{Especificación de Requisitos}

\section{Introducción}
Este anexo se encarga de mostrar los objetivos generales de la aplicación web, además de detallar los requisitos funcionales y no funcionales.

\section{Objetivos}
	El principal objetivo del proyecto es realizar un diseño web, en el cual se puedan ver los árboles singulares de las zonas universitarias de Burgos, con sus principales características.

	A través de un mapa podremos ver donde están ubicados los árboles. 
	\subsection{Objetivos Generales}
\begin{itemize}
	\item Observar en un mapa los árboles singulares de las zonas universitarias de Burgos y ver sus características.
	\item Filtrar a través de la familia, nombre científico y nombre común de la especie, autóctona y motivo singular donde se sitúan los árboles buscados y sus características.
	\item Loguearnos como usuario y ser capaces de importar y descargar datos de los árboles.
	\item Realizar una primera toma de contacto con la búsqueda de árboles, que en una futura mejora no solo busquemos arboles, es decir, que seamos capaces de buscar monumentos, lugares importantes \ldots
	\item Poder compartir y dar me gusta a la página de facebook de UbuVerde e interactuar con ellos a través de twitter. 
\end{itemize}

\subsection{Objetivos Técnicos}
\begin{itemize}
	\item Ser capaz de introducir datos en la base de datos Sqlite3.
	\item Ser capaz de descargar datos de la base de datos Sqlite3.
	\item Plasmar en el mapa esos datos introducidos en la base de datos. 
	\item Programar en Python y html el diseño web que va a tener nuestro proyecto.
	\item Guardar en un repositorio de GitHub los cambios que hemos ido realizando.
	\item Utilizar el framework Django para realizar correctamente nuestro diseño web.
	
\end{itemize}

\section{Catalogo de requisitos}
\subsection{Requisitos funcionales}
\begin{itemize}
	\item \textbf{RF-1 Cargar datos}: Los usuarios deben ser capaces de introducir datos en la base de datos y poder visualizarlos después.
	\item \textbf{RF-2 Exportar datos}: Los usuarios deben ser capaces de exportar datos de la base de datos, para poder guardar la información sin necesidad de visitar la página.
	\item \textbf{RF-3 Visualizar datos}: Los usuarios deben ser capaces de ver en un mapa los distintos árboles situados en las zonas universitarias.
	\subitem \textbf{RF-3.1}: Se podrán filtrar los árboles a través de la familia, nombre científico y común de la especie, si es autóctono, motivo singular \ldots
	\subitem \textbf{RF-3.2}: Se podrán buscar por los usuarios más expertos en la materia el árbol en concreto a través del nombre.
	\item \textbf{RF-4 Control de usuarios}: La aplicación debe ser capaz de tener controlado en todo momento al usuario logueado.
	\subitem \textbf{RF-4.1 Registro}: Los usuarios pueden registrarse para tener acceso a la descarga de los datos, así como su subida.
	\subitem \textbf{RF-4.2 Iniciar sesión}: Los usuarios pueden iniciar sesión con una cuenta previamente registrada.
	\subitem \textbf{RF-4.3 Cierre de sesión}: Los usuarios una vez terminada su visita en la página pueden cerrar sesión. 
	\item \textbf{RF-5 Búsqueda de datos}: Los usuarios deben ser capaces de buscar datos de un árbol específico y ver donde está situado.
	\item \textbf{RF-6 Aspecto visual}: Los usuarios deben ser capaces de visualizar las distintas pantallas de la aplicación en varios dispositivos con distintos tamaños y que el aspecto visual siga mostrándose de igual calidad. 
\subsection{Requisitos no funcionales}
\begin{itemize}
	\item \textbf{RNF-1 Usabilidad}: La aplicación web debe ser intuitiva y de fácil manejo para el usuario.
	\item \textbf{RNF-2 Mantenibilidad}: La aplicación web debe ser capaz de añadir nuevos datos.
	\item \textbf{RNF-3 Compatibilidad}: La aplicación web debe poder visualizarse en los principales navegadores, así como en los dispositivos móviles.
	\item \textbf{RNF-4 Rendimiento}: La aplicación web debe cargar los datos y mapas con una velocidad adecuada.
	\item \textbf{RNF-5 Responsividad}: La aplicación web debe poder visualizarse sin perder calidad y adaptarse al tamaño en los principales navegadores, así como en los dispositivos móviles. 
	\item \textbf{RNF-6 Escalabilidad}: A un mayor incremento de recursos la aplicación web debe ser capaz de incrementar en consecuencia su rendimiento.
	\item \textbf{RNF-7 Desplegabilidad}: La aplicación web debe ser capaz de intregarse en un servidor sin ningún problema.
	\end{itemize}
\end{itemize}

\section{Especificación de requisitos}

Este apartado será el encargado de mostrar los diagramas de casos de uso basados en los requisitos funcionales del proyecto, para ello se describirán tanto en forma de tabla como de diagrama. Ver figura \ref{fig:requisitos/casosUso}

\subsection{Diagrama de Casos de Uso}
\imagen{requisitos/casosUso}{Diagrama de Casos de Uso}

\subsection{Descripción de Casos de Uso}

A continuación se mostrará una tabla para cada uno de los casos de uso.
\newpage

% Caso de uso 1

\TablaCasoDeUso{1}{Cargar datos}
{RF-1}
{Carga datos en la base de datos para poder visualizarlos después.}
{El usuario abre el navegador, carga la página de la aplicación y se registre.}
{
	\item El usuario inicie sesión.
	\item El usuario introduce los datos del árbol que desea añadir.
}
{La información introducida es supervisada por el administrador y si es correcta se carga en la base de datos.}
{
	\item La información introducida no es correcta.
	\item La información introducida es correcta pero no se puede catalogar como árbol singular.
}
{Alta}

\newpage

% Caso de uso 2

\TablaCasoDeUso{2}{Exportar datos}
{RF-2}
{Exportar datos desde base de datos para poder guardar esa información.}
{El usuario abre el navegador, carga la página de la aplicación y se registre.}
{
	\item El usuario inicie sesión.
	\item El usuario selecciona los árboles que desea exportar.
	\item El usuario exporta los datos de los árboles deseados.
}
{Si el usuario está registrado podrá descargar los datos.}
{
	\item Si el usuario no está registrado no le dejará descargar los datos.
}
{Media}

\newpage

% Caso de uso 3

\TablaCasoDeUso{3}{Visualizar datos}
{RF-3, RF-3.1, RF-3.2}
{Visualizar en un mapa los árboles situados en las zonas universitarias.}
{El usuario abre el navegador, carga la página de la aplicación y que haya árboles introducidos.}
{
	\item Inicia la aplicación web y podrá ver los árboles más cercanos a su punto de partida.
	\item Si lo desea el usuario podrá filtrar los árboles por familia, nombre científico y común de la especie, si es autóctono, motivo singular y ver en el mapa los seleccionados.
	\item Si lo desea el usuario puede filtrar los árboles únicamente por el nombre y ver en el mapa su selección.
}
{Se muestra en el mapa los árboles seleccionados.}
{
	\item Error si los datos seleccionados no coinciden con ningún dato de la base de datos.
}
{Alta}

\newpage

% Caso de uso 4

\TablaCasoDeUso{4}{Controlar usuarios}
{RF-4, RF-4.1, RF-4.2, RF-4.3}
{Controlar a los usuarios que desean visitar la página web.}
{El usuario abre el navegador y carga la página de la aplicación.}
{
	\item El usuario pulsa el botón de registrarse.
	\item El usuario introduce los datos para registrarse.
	\item El usuario pulsa el botón de iniciar sesión.
	\item El usuario introduce los datos para iniciar sesión.
	\item El usuario pulsa el botón de cerrar sesión.
}
{Redirecciona a la página con la sesión iniciada.
 Redirecciona a la página con la sesión cerrada.}
{
	\item Error si los datos introducidos al registrarse no son correctos o ya existen.
	\item Error si al iniciar sesión los datos introducidos no existen.
}
{Alta}

\newpage

% Caso de uso 5

\TablaCasoDeUso{5}{Buscar datos específicos}
{RF-5}
{Buscar un árbol específico para ver donde está situado.}
{El usuario abre el navegador y carga la página de la aplicación.}
{
	\item El usuario introduce las características del árbol que desea buscar.
}
{Se muestra en el mapa el árbol seleccionado.}
{
	\item Error si los datos introducidos del árbol no existen.
}
{Media}

\newpage

% Caso de uso 6

\TablaCasoDeUso{6}{Visualizado en varios dispositivos}
{RF-6}
{Ver en distintos dispositivos la página web sin perder calidad.}
{El usuario abre el navegador y carga la página de la aplicación.}
{
	\item El usuario abre la página web con el móvil, tablet, distintos navegadores de ordenador.
}
{La página web se ve correctamente.}
{
	\item Error si la página web no se ve correctamente o pierde calidad.
}
{Media}

\newpage
	\apendice{Especificación de diseño}

\section{Introducción}

En este apartado explicaremos el diseño que ha dado lugar a la aplicación, para ello el anexo está dividido en tres secciones: el diseño de datos, el diseño procedimental y el diseño arquitectónico.

\section{Diseño de datos}

Para el almacenaje de los datos de la aplicación he utilizado una base de datos sqlite3, la cual está compuesta por seis tablas o modelos.

\imagen{diseño/familias}{Tabla Familia}

\begin{itemize}
	\item \textbf{FAMILIA}: En este modelo se va a guardar la información relacionada con las familias de árboles existentes en las universidades de Burgos, está compuesta por dos campos que son:
	\begin{itemize}
		\item \textbf{idFamilia}: Es la clave primaria de la tabla y contiene un identificador único para cada familia.
		\item \textbf{nombreFamilia}: Contiene el nombre de la familia, es un campo de tipo texto, tiene que ser único, no se puede dejar en blanco y con un tamaño máximo de 30 caracteres. 
	\end{itemize}
\end{itemize}

\imagen{diseño/generos}{Tabla Genero}

\begin{itemize}
	\item \textbf{GENERO}: En este modelo se va a guardar la información relacionada con los géneros de árboles existentes en las universidades de Burgos, está compuesta por tres campos que son:
	\begin{itemize}
		\item \textbf{idGenero}: Es la clave primaria de la tabla y contiene un identificador único para cada género.
		\item \textbf{nombreFamilia}: Contiene el nombre del género, es un campo de tipo texto, tiene que ser único, no se puede dejar en blanco y con un tamaño máximo de 30 caracteres. 
		\item \textbf{familia}: Contiene el nombre de la familia, es una FK.
	\end{itemize}
\end{itemize}

\imagen{diseño/especies}{Tabla Especie}

\begin{itemize}
	\item \textbf{ESPECIE}: En este modelo se va a guardar la información relacionada con las especies de árboles existentes en las universidades de Burgos, está compuesta por siete campos que son:
	\begin{itemize}
		\item \textbf{idEspecie}: Es la clave primaria de la tabla y contiene un identificador único para cada especie.
		\item \textbf{nombreCientificoEspecie}: Contiene el nombre científico de la especie, es un campo de tipo texto, no se puede dejar en blanco y con un tamaño máximo de 50 caracteres.
		\item \textbf{nombreComunEspecie}: Contiene el nombre común de la especie, es un campo de tipo texto, no se puede dejar en blanco y con un tamaño máximo de 50 caracteres.
		\item \textbf{genero}: Contiene el nombre del género, es una FK.
		\item \textbf{autoctona}: Contiene si la especie es autóctona, es un campo de tipo booleano y no se puede dejar en blanco.
		\item \textbf{descripcion}: Contiene una pequeña descripción de la especie, es un campo de tipo texto y no se puede dejar en blanco.
		\item \textbf{ecologia}: Contiene la ecología de la especie, es un campo de tipo texto y no se puede dejar en blanco.
	\end{itemize}
\end{itemize}

\imagen{diseño/individuos}{Tabla Individuos}

\begin{itemize}
	\item \textbf{INDIVIDUO}: En este modelo se va a guardar la información relacionada con cada uno de árboles existentes en las universidades de Burgos, está compuesta por catorce campos que son:
	\begin{itemize}
		\item \textbf{idIndividuo}: Es la clave primaria de la tabla y contiene un identificador único para cada individuo.
		\item \textbf{nombreComun}: Contiene el nombre común del árbol, es un campo de tipo texto, no se puede dejar en blanco y con un tamaño máximo de 30 caracteres. 
		\item \textbf{especie}: Contiene el nombre de la especie, es una FK.
		\item \textbf{motivoSingular}: Contiene un motivo de por qué el árbol es singular, es un campo de tipo texto, no se puede dejar en blanco y con un tamaño máximo de 50 caracteres. 
		\item \textbf{explicacionMotivoSingular}: Contiene una explicación más detallada de porque el árbol es singular, es un campo de tipo texto y no se puede dejar en blanco.
		\item \textbf{latitud}: Contiene la latitud del árbol, es un campo de tipo decimal y no se puede dejar en blanco.
		\item \textbf{longitud}: Contiene la longitud del árbol, es un campo de tipo decimal y no se puede dejar en blanco.
		\item \textbf{fotoArbol}: Contiene la foto del árbol, es un campo de tipo imagen y si se puede dejar en blanco.
		\item \textbf{fotoHojas}: Contiene la foto de las hojas del árbol, es un campo de tipo imagen y si se puede dejar en blanco.
		\item \textbf{fotoTronco}: Contiene la foto del tronco del árbol, es un campo de tipo imagen y si se puede dejar en blanco.
		\item \textbf{fotoFrutos}: Contiene la foto de los frutos del árbol, es un campo de tipo imagen y si se puede dejar en blanco.
		\item \textbf{ubicacion}: Contiene la ubicación del árbol, es decir, la zona universitaria donde está ubicado, es un campo de tipo texto, no se puede dejar en blanco y con un tamaño máximo de 50 caracteres.
		\item \textbf{altura}: Contiene la altura del árbol, es un campo de tipo decimal, si se puede dejar en blanco y con un tamaño máximo de 19 caracteres.
		\item \textbf{perimetro}: Contiene el perímetro del árbol, es un campo de tipo decimal, si se puede dejar en blanco y con un tamaño máximo de 19 caracteres.  
	\end{itemize}
\end{itemize}

\imagen{diseño/usuarios}{Tabla Usuario}

\begin{itemize}
	\item \textbf{USUARIO}: En este modelo se va a guardar la información relacionada con los usuarios, está compuesta por dos campos que son:
		\begin{itemize}
		\item \textbf{idUsuario}: Es la clave primaria de la tabla y contiene un identificador único para cada usuario.
		\item \textbf{usuario}: Contiene el nombre del usuario, es una FK. 
	\end{itemize}
\end{itemize}

\begin{itemize}
	\item \textbf{USER}: Este modelo está diseñado por el framework y guarda información relacionada con los usuarios, está compuesta por doce campos que son:
	\begin{itemize}
		\item \textbf{username}: Contiene el nombre del usuario, es un campo de tipo texto, tiene que ser único, no se puede dejar en blanco y con un tamaño máximo de 150 caracteres.
		\item \textbf{first name}: Contiene el nombre del usuario, es un campo de tipo texto, se puede dejar en blanco y con un tamaño máximo de 30 caracteres.
		\item \textbf{last name}: Contiene el apellido del usuario, es un campo de tipo texto, se puede dejar en blanco y con un tamaño máximo de 150 caracteres.  
		\item \textbf{email}: Contiene el email del usuario, es un campo de tipo email y se puede dejar en blanco.
		\item \textbf{password}: Contiene la contraseña del usuario, es un campo de tipo texto, es necesario y se encripta.
		\item \textbf{groups}: Contiene el grupo al cual pertenece el usuario. 
		\item \textbf{user permissions}: Contiene los permisos que tiene cada usuario.
		\item \textbf{is staff}: Nos indica si este usuario puede acceder al sitio de administración.
		\item \textbf{is active}: Nos indica si este usuario está activo.  
		\item \textbf{is superuser}: Nos indica si este usuario es super usuario.
		\item \textbf{last login}: Nos indica una fecha y hora del último inicio de sesión del usuario.
		\item \textbf{date joined}: Nos indica una fecha y hora de cuando se creó la cuenta del usuario. 
	\end{itemize}
	
\end{itemize}

\section{Diseño procedimental}

Para comprender y entender las funciones que va a desempeñar nuestra aplicación web he realizado un diagrama de navegabilidad \ref{fig:diseño/diagramaNavegabilidad}, el cual explicaré en detalle a continuación:

\begin{itemize}
	\item En primer lugar el usuario accede a la pantalla principal de la página.
	\item Una vez dentro tiene dos opciones, registrarse e iniciar sesión o navegar sin las ventajas de ser usuario.
	\item Si el usuario no inicia sesión podrá:
	\begin{itemize}
		\item Seguir en Twitter a @UbuVerde.
		\item Dar Me Gusta y compartir noticias de UbuVerde. 
		\item Visitar la página de UbuVerde (Acerca De Nosotros). 
		\item Ver el mapa con todos los árboles y seleccionar uno y ver sus características y fotos.
		\item Ver todas las familias:
		\begin{itemize}
			\item Elegir una o ver el mapa y seleccionar un árbol.
			\item Si seleccionas una familia, la ves y eliges un género.
			\item Una vez seleccionado el género, ves el género, la familia y eliges la especie.
			\item Ves las características de la especie y eliges el individuo.
			\item Ves las fotos, el mapa y las características del árbol seleccionado.
			\item Si seleccionas un árbol ves las características del árbol, las fotos y el mapa donde esta ubicado.
		\end{itemize} 
		\item Ver todos los géneros:
		\begin{itemize}
			\item Elegir uno o ver el mapa y seleccionar un árbol.
			\item Si seleccionas un género, ves la familia y el género seleccionado y eliges una especie.
			\item Ves las características de la especie y eliges el individuo.
			\item Ves las fotos, el mapa y las características del árbol seleccionado.
			\item Si seleccionas un árbol ves las características del árbol, las fotos y el mapa donde esta ubicado. 
		\end{itemize}
		\item Ver todas las especies:
		\begin{itemize}
			\item Elegir una o ver el mapa y seleccionar un árbol.
			\item Si seleccionas una especie, ves las características de la especie y eliges el individuo.
			\item Ves las fotos, el mapa y las características del árbol seleccionado.
			\item Si seleccionas un árbol ves las características del árbol, las fotos y el mapa donde esta ubicado.
		\end{itemize}
		\item Ver todos los individuos:
		\begin{itemize}
			\item Ves las 5 ubicaciones y seleccionas una.
			\item Ves el mapa con los árboles de esa ubicación y seleccionas uno.
			\item Ves las fotos, el mapa y las características del árbol seleccionado.
		\end{itemize}
	\end{itemize}
	\item Si el usuario inicia sesión podrá además de todas esas cosas anteriores:
	\begin{itemize}
		\item Descargar un pdf con las características de los árboles.
		\item Agregar nuevos árboles.
		\item Cerrar sesión.
	\end{itemize}
	
\end{itemize}

\imagen{diseño/diagramaNavegabilidad}{Diagrama de Navegabilidad}

\section{Diseño arquitectónico}

El proyecto como está realizado mediante el framework Django seguirá el patrón MVT\cite{MVT}.

Es un framework muy parecido al MVC, pero con ligeros matices:
\begin{itemize}
	\item \textbf{M}: Significa Model o Modelo, es la capa que tiene acceso a la base de datos. En ella esta contenida la información relacionada con los datos, es decir, cómo acceder a ellos, cómo se comportan, cómo se validan y las relaciones entre sí.
	\item \textbf{V}: Significa View o Vista, es la capa de la lógica de negocios. En ella esta contenida la lógica que accede al modelo y se comunica con el template o plantilla deseada. 
	\item \textbf{T}: significa Template o Plantilla, es la capa de presentación. En ella estará contenido las decisiones que tienen que ver con la presentación de los datos, en nuestro caso como se va a ver la página web.
\end{itemize}

Muchas veces se tiende a decir que las Vistas de Django se pueden asemejar al controlador y las Plantillas pueden ser las vistas en el MVC, pero esto no es cierto, en realidad la vista en MVC describe los datos que son presentados al usuario; no necesariamente el cómo se mostrarán, pero si cuáles datos son presentados.
	\apendice{Documentación técnica de programación}

\section{Introducción}

En este apartado se va a explicar todo lo que tiene que conocer un programador para seguir con el desarrollo del proyecto, la estructura de directorios, la compilación, instalación y ejecución del proyecto y las pruebas realizadas.

\section{Estructura de directorios}

En este apartado se van a describir y poner en manifiesto la estructura de directorios del proyecto:

\begin{itemize}
	\item \textbf{TFG ARBUBU/}: Es la carpeta raíz, donde está contenido la documentación y la parte de programación del proyecto.
	\begin{itemize}
		\item \textbf{/Latex/}: Es la carpeta donde está la documentación del proyecto:
		\begin{itemize}
			\item \textbf{/Latex/memoria.pdf}: Es el documento que contiene la memoria del proyecto en formato .pdf
			\item \textbf{/Latex/anexo.pdf}: Es el documento que contiene los anexos del proyecto en formato .pdf
			\item \textbf{/Latex/img/}: Es la carpeta que contiene las imágenes utilizadas en el documento latex.
			\item \textbf{/Latex/text/}: Es la carpeta que contiene los documentos .text utilizados tanto en la memoria como en los anexos.
		\end{itemize}
		\item \textbf{/Proyectos/}: Es la carpeta donde está alojado tanto el entorno virtual como la parte de programación del proyecto:
		\begin{itemize}
			\item \textbf{/Proyectos/Entorno}: Es la carpeta donde está ubicado el entorno virtual del proyecto.
			\item \textbf{/Proyectos/arbubu/}: Es la carpeta donde está ubicado la programación del proyecto:
			\begin{itemize}
				\item \textbf{/Proyectos/arbubu/aplicaciones/}: Es la parte del código que separa una aplicación de otra, en mi caso solo tengo una, pero si tuviera más iría en este apartado y serían aplicaciones totalmente independientes.
				\item \textbf{/Proyectos/arbubu/aplicaciones/principal/}: Contiene los distintos archivos de la aplicación:
				\item \textbf{/Proyectos/arbubu/aplicaciones/principal/init.py}: Es un archivo vacío le dice a Python que esta carpeta es un paquete.
				\item \textbf{/Proyectos/arbubu/aplicaciones/principal/admin.py}: Es un archivo de que contiene la configuración para poder usar la aplicación Django Admin.
				\item \textbf{/Proyectos/arbubu/aplicaciones/principal/apps.py}: Es un archivo que contiene la configuración de nuestra aplicación.
				\item \textbf{/Proyectos/arbubu/aplicaciones/principal/forms.py}: Es un archivo que contiene la configuración de los formularios que vamos a utilizar en nuestra aplicación.
				\item \textbf{/Proyectos/arbubu/aplicaciones/principal/models.py}: Es un archivo que contiene las entidades que vamos a definir en nuestra aplicación web, cada uno de los modelos definidos se convierten en una tabla en nuestra base de datos. 
				\item \textbf{/Proyectos/arbubu/aplicaciones/principal/urls.py}: Es un archivo que contiene todas las urls definidas en nuestra aplicación.
				\item \textbf{/Proyectos/arbubu/aplicaciones/principal/views.py}: Es un archivo que contiene todas las vistas definidas en nuestra aplicación.
				\item \textbf{/Proyectos/arbubu/arbubu/}: Es la parte del código que engloba a todas las aplicaciones y tiene las configuraciones globales.
				\item \textbf{/Proyectos/arbubu/arbubu/init.py}: Es un archivo vacío le dice a Python que esta carpeta es un paquete.
				\item \textbf{/Proyectos/arbubu/arbubu/settings.py}: Es un archivo que contiene todos los ajustes, registramos las aplicaciones que hemos creado, donde se ubican los ficheros estáticos, la configuración de la base de datos...
				\item \textbf{/Proyectos/arbubu/arbubu/urls.py}: Es un archivo que contiene las urls de todas las aplicaciones creadas y del paquete de administración.
				\item \textbf{/Proyectos/arbubu/arbubu/wsgi.py}: Es una archivo que ayuda a Django a comunicarse con el servidor web.
				\item \textbf{/Proyectos/arbubu/static/}: Es un archivo que contiene los archivos estáticos del proyecto.
				\item \textbf{/Proyectos/arbubu/static/css}: Contiene todas las hojas de estilos .css
				\item \textbf{/Proyectos/arbubu/static/fonts}: Contiene los archivos fuente de los slides.
				\item \textbf{/Proyectos/arbubu/static/imagenes}: Contiene las imágenes que vamos a utilizar en nuestra aplicación.
				\item \textbf{/Proyectos/arbubu/static/individuosJs}: Contiene los archivos .js de cada uno de nuestros árboles.
				\item \textbf{/Proyectos/arbubu/static/js}: Contiene los archivos .js de los slides.
				\item \textbf{/Proyectos/arbubu/templates/}: Contiene todos los templates.
				\item \textbf{/Proyectos/arbubu/templates/principal}: Contiene todos los templates de la aplicación "principal".
				\item \textbf{/Proyectos/arbubu/templates/principal/especies}: Contiene los html de cada una de las especies.
				\item \textbf{/Proyectos/arbubu/templates/principal/familias}: Contiene los html de cada una de las familias.
				\item \textbf{/Proyectos/arbubu/templates/principal/generos}: Contiene los html de cada una de los géneros.
				\item \textbf{/Proyectos/arbubu/templates/principal/individuos}: Contiene los html de cada una de los individuos.
				\item \textbf{/Proyectos/arbubu/templates/principal/addIndividuo.html}: Es el archivo .html que permite añadir un nuevo individuo.
				\item \textbf{/Proyectos/arbubu/templates/principal/especies.html}: Es el archivo .html que contiene las llamadas a cada una de las especies.
				\item \textbf{/Proyectos/arbubu/templates/principal/familias.html}: Es el archivo .html que contiene las llamadas a cada una de las familias.
				\item \textbf{/Proyectos/arbubu/templates/principal/generos.html}: Es el archivo .html que contiene las llamadas a cada uno de los géneros.
				\item \textbf{/Proyectos/arbubu/templates/principal/index.html}: Es el archivo .html que contiene la pantalla principal de la aplicación.
				\item \textbf{/Proyectos/arbubu/templates/principal/individuos.html}: Es el archivo .html que contiene las llamadas a cada una de las ubicaciones de los individuos.
				\item \textbf{/Proyectos/arbubu/templates/principal/individuos-ciencias.html}: Es el archivo .html que contiene las llamadas a cada una de los individuos de la facultad de ciencias.
				\item \textbf{/Proyectos/arbubu/templates/principal/individuos-educacion.html}: Es el archivo .html que contiene las llamadas a cada una de los individuos de la facultad de educación.
				\item \textbf{/Proyectos/arbubu/templates/principal/individuos-hospital-militar.html}: Es el archivo .html que contiene las llamadas a cada una de los individuos del hospital militar.
				\item \textbf{/Proyectos/arbubu/templates/principal/individuos-hospital-del-rey.html}: Es el archivo .html que contiene las llamadas a cada una de los individuos del hospital del rey.
				\item \textbf{/Proyectos/arbubu/templates/principal/individuos-rio-vena.html}: Es el archivo .html que contiene las llamadas a cada una de los individuos de la facultad de río vena.
				\item \textbf{/Proyectos/arbubu/templates/principal/iniciar-sesion.html}: Es el archivo .html que permite iniciar sesión a un usuario.
				\item \textbf{/Proyectos/arbubu/templates/principal/usuario-form.html}: Es el archivo .html que permite registrarse a un usuario.
				\item \textbf{/Proyectos/arbubu/db.sqlite3}: Es el archivo que guarda la base de datos de nuestro proyecto.
				\item \textbf{/Proyectos/arbubu/manage.py}: Es un archivo usado para ejecutar comandos de administración relacionados con nuestro proyecto.
			\end{itemize}
		\end{itemize}
	\end{itemize}
\end{itemize}
\section{Manual del programador}

Este apartado voy a explicar las cosas que he tenido que instalar para que en un futuro, si alguien quiere mejorar la aplicación web, solo tenga que seguir los pasos aquí y marcados y seguir adelante con ello.

\begin{itemize}
	\item \textbf{Atom}: Para poder modificar el código de la aplicación lo primero que tenemos que hacer es descargar este programa \footnote{https://atom.io/}, una vez que ha sido descargado cogemos el código fuente del proyecto de la siguiente dirección \cite{Repositorio} y empezamos a instalar las librerías necesarias.
	\imagen{manualProgramador/atom}{Descarga Atom} 
	
	\item \textbf{Instalar Django}: Para poder utilizar Django primero tenemos que instalarlo.
	\imagen{manualProgramador/instalarDjango}{Instalar Django}
	
	\item \textbf{Python}: Para poder manejar el código tenemos que tener instalado Python \footnote{https://www.python.org/downloads/}.
	\imagen{manualProgramador/python}{Descarga Python}
	
	\item \textbf{Librerías}: En nuestro caso con coger el código fuente de la dirección anteriormente citada nos valdría, pero si quieres empezar de cero un proyecto debes realizar lo siguiente:
	
		\begin{itemize}
			\item \textbf{Crear Entorno Virtual}: En caso de que surja algun error en tu proyecto, solo necesitarás borrar el entorno virtual y volver a instalarlo, en cambio si no se usa entorno virtual puede que sea necesario volver a instalar todo el proyecto.
			\imagen{manualProgramador/entornoVirtual}{Crear Entorno Virtual}
			
			\item \textbf{Crear Proyecto Django}: Crear un proyecto de django de la siguiente forma: 
			\imagen{manualProgramador/crearProyectoDjango}{Crear Proyecto Django}
			
			\item \textbf{Crear Aplicación}: Crear distintas aplicaciones, para que así estén separadas unas de otras y sean totalmente independientes.
			\imagen{manualProgramador/crearAplicacionPrincipal}{Crear Aplicación}
			
			\item \textbf{Crear Super Usuario}: Para poder manejar el sitio de administración hay que crear un super usuario.
			\imagen{manualProgramador/crearUsuario}{Crear Super Usuario}
			
			\item \textbf{Instalar Pillow}: Es una librería que permite introducir fotos en la base de datos y poder trabajar con estos archivos.
			\imagen{manualProgramador/instalarPillow}{Instalar Pillow}
			
		\end{itemize}
\end{itemize}

\section{Compilación, instalación y ejecución del proyecto}

\subsection{Compilación e instalación}

Como hemos comentado anteriormente no es necesario compilar e instalar nada, lo único que se necesita es descargar el código fuente de la siguiente dirección \cite{Repositorio}.

\subsection{Ejecución del proyecto}

\begin{itemize}
	\item \textbf{Activar entorno virtual}: Para trabajar con el entorno virtual creado anteriormente tenemos que activarlo y se hace así:
	\imagen{manualProgramador/activarEntornoVirtual}{Activar Entorno Virtual}
	
	\item \textbf{Ejecutar proyecto}: Para poder entrar en nuestra aplicación web tenemos que hacer lo siguiente:
	\imagen{manualProgramador/ejecutarProyecto}{Ejecutar proyecto}
	
	Una vez ejecutado ese comando nos dirigimos a cualquier navegador y entramos a la dirección localhost que nos muestra en la pantalla de comandos que es: http://127.0.0.1:8000 y le añadimos la pagina principal que es index, por lo que debemos ir a la siguiente dirección: http://127.0.0.1:8000/index y una vez dentro ya podemos movernos y realizar las funcionalidades de la aplicación.
\end{itemize}

	\apendice{Documentación de usuario}

\section{Introducción}

En este apartado explicaremos los requisitos mínimos que debe cumplir el dispositivo donde se ejecutará el diseño web, lo que deben instalar y un manual de usuario.

\section{Requisitos de usuarios}

Al tratarse de una aplicación web, los requisitos que necesita el usuario son:

\begin{itemize}
	\item Un navegador instalado (Google Chrome, Microsoft Edge, Mozilla Firefox, Opera ...)
	\item Cookies activadas.
	\item Compatibilidad con hojas de estilo CSS.
	\item JavaScript activado.
	\item Conexión a internet, ya que los mapas lo necesitan para cargarse.
	\item Al ejecutarse en localhost, necesitará el código fuente y el entorno virtual creado.
\end{itemize}
 
 La aplicación web se ha probado en diferentes navegadores y funciona correctamente, también funcionaría para dispositivos móviles ya que por lo único que nos podría dar algún problema es la introducción de mapas y la tecnología leaflet está pensada para que funcione en el móvil correctamente.

\section{Instalación}

Al trabajar en localhost como hemos explicado anteriormente, necesitaremos disponer del código fuente que se podrá descargar desde GitHub \cite{GitHub}. El repositorio se podrá descargar a partir de esta dirección: \cite{Repositorio}

Como se trata de un manual para que el usuario sea capaz de probar la aplicación web y no de manipular el código no será necesario la instalación de ningún componente extra.

\section{Manual del usuario}

En este apartado se mostrará como manejar la aplicación. Como hemos explicado anteriormente no es necesario registrarse, ni iniciar sesión para navegar en la página, simplemente agregará funcionalidades. 

\subsection{Registrarse}

Una vez dentro de la dirección de la aplicación nos encontraremos con esta página inicial \imagen{manualUsuario/portadaSinRegistrar}{PortadaSinregistro},en la parte de arriba a la derecha tenemos tres botones, INICIA SESION, REGISTRATE y ACERCA DE NOSOTROS. En este caso pulsaremos sobre REGISTRATE.

Nos encontraremos con esta página \imagen{manualUsuario/registrarse}{Registro} y cumplimentamos los campos. Una vez registrados nos mandará a la página de inicio y automáticamente iniciará sesión. \imagen{manualUsuario/portadaConRegistro}{PortadaConregistro}

\subsection{Iniciar Sesión}

Si ya estamos registrados, unicamente tendremos que iniciar sesión, nos encontraremos con esta pantalla: \ref{fig:manualUsuario/portadaSinRegistrar} para ello como dijimos anteriormente nos encontraremos los tres botones, pero en este caso iremos a INICIA SESION.

Veremos esta página \imagen{manualUsuario/iniciaSesion}{Iniciar Sesión} y cumplimentamos los campos. Una vez iniciada la sesión nos mandará a la página de inicio. \ref{fig:manualUsuario/portadaConRegistro}

\subsection{Seguir a @UbuVerde, compartir y dar Like en Facebook}

En la página de inicio \ref{fig:manualUsuario/portadaSinRegistrar} arriba a la derecha debajo de los tres botones mencionados tendremos un botón de seguir a @ubuVerde en twitter y dos botones seguidos de compartir y dar me gusta en Facebook a la página de UbuVerde.

\subsection{Visitar página UbuVerde}

Como dijimos anteriormente en la página de inicio \ref{fig:manualUsuario/portadaSinRegistrar} arriba a la derecha el último botón es ACERCA DE NOSOTROS, si pulsamos en el nos mandará a la página de UbuVerde.
\imagen{manualUsuario/paginaUbuVerde}{Página UbuVerde} 

\subsection{Descargar Individuos}

En cambio si sí nos registramos e iniciamos sesión tendremos la posibilidad de descargarnos un pdf con las caracteristicas de todos los árboles y su ubicación, nos encontramos en esta pantalla: \ref{fig:manualUsuario/portadaConRegistro}

 
Ahora ya no tendremos arriba a la derecha los botones de iniciar sesión y registrarse, si no que aparecerán el botón de CERRAR SESION, DESCARGAR INDIVIDUOS, AGREGAR INDIVIDUOS y ACERCA DE NOSOTROS.


Si pulsamos en DESCARGAR INDIVIDUOS, nos abrirá un pdf que podremos descargar. \imagen{manualUsuario/pdf}{PDF árboles}

\subsection{Agregar Individuos}

En la pantalla de inicio \ref{fig:manualUsuario/portadaConRegistro} si estamos registrados, en la parte de arriba a la derecha podemos introducir algún árbol en la base de datos si pulsamos en AGREGAR INDIVIDUOS. Nos redirigirá a esta página \imagen{manualUsuario/agregarArbol}{Agregar árboles} cumplimentamos los campos y automáticamente se agregará a la base de datos.

\subsection{Ver Mapa}

En la pantalla de inicio independientemente estemos o no registrados, veremos en la parte central un mapa con todos los árboles que he introducido en la base de datos,\imagen{manualUsuario/arbolMapa}{Arbol resaltado en mapa} si pulsamos en alguno de los iconos se verá resaltado las características de dicho árbol y un enlace \ref{fig:manualUsuario/pantallaArbol} para ver más en profundidad esas características, además de sus fotos y el mapa con el árbol resaltado.

\subsection{Ver Arbol}

En la pantalla de inicio como hemos dicho antes en la parte central tenemos un mapa y si pinchamos en algún icono se vera resaltado un árbol con sus características y un enlace, si pulsamos en dicho enlace nos llevará a esta página \ref{fig:manualUsuario/pantallaArbol} donde podremos observar las características del árbol y un slide a la derecha con las fotos y el mapa donde está ubicado el árbol.
\imagen{manualUsuario/pantallaArbol}{Pantalla Arbol}

\subsection{Ver Familias}

En la pantalla de inicio independientemente estemos o no registrados, veremos un menú en el cual tendremos INICIO, FAMILIAS, GENEROS, ESPECIES e INDIVIDUOS, en este caso debemos ir a FAMILIAS. Una vez pulsado nos aparecerá esta pantalla: 
\imagen{manualUsuario/pantallaFamilia}{Pantalla Familias}

en ella veremos una lista desplegable con todas las familias existentes y un mapa con todos los árboles, si pulsamos en cualquier icono del mapa y pinchamos en el enlace iremos a esta página \ref{fig:manualUsuario/pantallaArbol} y si pulsamos en alguna de las familias existentes en ir nos llevará a la siguiente pantalla:
\imagen{manualUsuario/pantallaFamilia2}{Segunda Pantalla Familias}

en ella veremos la familia seleccionada y una lista desplegable de todos los géneros de dicha especie, si elegimos uno y pulsamos en ir nos aparecerá la siguiente pantalla:
\imagen{manualUsuario/pantallaFamilia3}{Tercera Pantalla Familias}

en ella veremos la familia y el genero seleccionados y una lista desplegable con todas las especies de dicho genero, si elegimos uno y pulsamos en ir nos aparecerá la siguiente pantalla:
\imagen{manualUsuario/pantallaFamilia4}{Cuarta Pantalla Familias}

en ella veremos las características de la especie seleccionada y una lista desplegable con todos los árboles de la especie, si elegimos uno y pulsamos en ir nos aparecerá la siguiente pantalla: \ref{fig:manualUsuario/pantallaArbol}

\subsection{Ver Géneros}

En la pantalla de inicio independientemente estemos o no registrados, veremos un menú en el cual tendremos INICIO, FAMILIAS, GENEROS, ESPECIES e INDIVIDUOS, en este caso debemos ir a GENEROS. Una vez pulsado nos aparecerá esta pantalla: 
\imagen{manualUsuario/pantallaGenero}{Pantalla Géneros}

en ella veremos una lista desplegable con todos los géneros existentes y un mapa con todos los árboles, si pulsamos en cualquier icono del mapa y pinchamos en el enlace iremos a esta página \ref{fig:manualUsuario/pantallaArbol} y si pulsamos en alguno de los géneros existentes en ir nos llevará a la siguiente pantalla: \ref{fig:manualUsuario/pantallaFamilia3}

en ella veremos el genero seleccionado y la familia a la que pertenece y una lista desplegable con todas las especies de dicho genero, si elegimos uno y pulsamos en ir nos aparecerá la siguiente pantalla: \ref{fig:manualUsuario/pantallaFamilia4}

en ella veremos las características de la especie seleccionada y una lista desplegable con todos los árboles de la especie, si elegimos uno y pulsamos en ir nos aparecerá la siguiente pantalla: \ref{fig:manualUsuario/pantallaArbol}

\subsection{Ver Especies}

En la pantalla de inicio independientemente estemos o no registrados, veremos un menú en el cual tendremos INICIO, FAMILIAS, GENEROS, ESPECIES e INDIVIDUOS, en este caso debemos ir a ESPECIES. Una vez pulsado nos aparecerá esta pantalla: 
\imagen{manualUsuario/pantallaEspecies}{Pantalla Especies}

en ella veremos una lista desplegable con todas las especies existentes y un mapa con todos los árboles, si pulsamos en cualquier icono del mapa y pinchamos en el enlace iremos a esta página \ref{fig:manualUsuario/pantallaArbol} y si pulsamos en alguna de las especies existentes en ir nos llevará a la siguiente pantalla: \ref{fig:manualUsuario/pantallaFamilia4}

en ella veremos las características de la especie seleccionada y una lista desplegable con todos los árboles de la especie, si elegimos uno y pulsamos en ir nos aparecerá la siguiente pantalla: \ref{fig:manualUsuario/pantallaArbol}

\subsection{Ver Individuos}

En la pantalla de inicio independientemente estemos o no registrados, veremos un menú en el cual tendremos INICIO, FAMILIAS, GENEROS, ESPECIES e INDIVIDUOS, en este caso debemos ir a INDIVIDUOS. Una vez pulsado nos aparecerá esta pantalla: 
\imagen{manualUsuario/pantallaIndividuos}{Pantalla Individuos}

en ella veremos las cinco ubicaciones y si pulsamos sobre cualquiera de ellas veremos:  \imagen{manualUsuario/pantallaIndividuos2}{Segunda Pantalla Individuos}

podemos observar como se resalta la ubicación seleccionada, una lista desplegable con todos los árboles que hay en esa ubicación y un mapa, si pulsamos en cualquier icono del mapa y pinchamos en el enlace iremos a esta página \ref{fig:manualUsuario/pantallaArbol} y si pulsamos en alguno de los arboles existentes en ir nos llevará también a la  misma pantalla: \ref{fig:manualUsuario/pantallaArbol}
	
	
	\bibliographystyle{plain}
	\bibliography{bibliografiaAnexos}
	
\end{document}