\documentclass[a4paper,12pt,twoside]{memoir}

% Castellano
\usepackage[spanish,es-tabla]{babel}
\selectlanguage{spanish}
\usepackage[utf8]{inputenc}
\usepackage[T1]{fontenc}
\usepackage{lmodern} % Scalable font
\usepackage{microtype}
\usepackage{placeins}

\RequirePackage{booktabs}
\RequirePackage[table]{xcolor}
\RequirePackage{xtab}
\RequirePackage{multirow}

% Links
\usepackage[colorlinks]{hyperref}
\hypersetup{
	allcolors = {red}
}

% Ecuaciones
\usepackage{amsmath}

% Rutas de fichero / paquete
\newcommand{\ruta}[1]{{\sffamily #1}}

% Párrafos
\nonzeroparskip


% Imagenes
\usepackage{graphicx}
\newcommand{\imagen}[2]{
	\begin{figure}[!h]
		\centering
		\includegraphics[width=0.9\textwidth]{#1}
		\caption{#2}\label{fig:#1}
	\end{figure}
	\FloatBarrier
}

\newcommand{\imagenPequena}[2]{
	\begin{figure}[!h]
		\centering
		\includegraphics[width=0.3\textwidth]{#1}
	\end{figure}
	\FloatBarrier
}
\newcommand{\insertarimagen}[3]{
	\begin{figure}[!h]
		\centering
		\includegraphics[width=0.9\textwidth]{#1}
		\caption[#2]{#2 \cite{#3}}\label{fig:#1}
	\end{figure}
	\FloatBarrier
}

\newcommand{\insertarimagenGrande}[3]{
	\begin{figure}[!h]
		\centering
		\includegraphics[width=0.4\textwidth]{#1}
		\caption[#2]{#2 \cite{#3}}\label{fig:#1}
	\end{figure}
	\FloatBarrier
}

\newcommand{\imagenflotante}[2]{
	\begin{figure}%[!h]
		\centering
		\includegraphics[width=0.9\textwidth]{#1}
		\caption{#2}\label{fig:#1}
	\end{figure}
}



% El comando \figura nos permite insertar figuras comodamente, y utilizando
% siempre el mismo formato. Los parametros son:
% 1 -> Porcentaje del ancho de página que ocupará la figura (de 0 a 1)
% 2 --> Fichero de la imagen
% 3 --> Texto a pie de imagen
% 4 --> Etiqueta (label) para referencias
% 5 --> Opciones que queramos pasarle al \includegraphics
% 6 --> Opciones de posicionamiento a pasarle a \begin{figure}
\newcommand{\figuraConPosicion}[6]{%
  \setlength{\anchoFloat}{#1\textwidth}%
  \addtolength{\anchoFloat}{-4\fboxsep}%
  \setlength{\anchoFigura}{\anchoFloat}%
  \begin{figure}[#6]
    \begin{center}%
      \Ovalbox{%
        \begin{minipage}{\anchoFloat}%
          \begin{center}%
            \includegraphics[width=\anchoFigura,#5]{#2}%
            \caption{#3}%
            \label{#4}%
          \end{center}%
        \end{minipage}
      }%
    \end{center}%
  \end{figure}%
}

%
% Comando para incluir imágenes en formato apaisado (sin marco).
\newcommand{\figuraApaisadaSinMarco}[5]{%
  \begin{figure}%
    \begin{center}%
    \includegraphics[angle=90,height=#1\textheight,#5]{#2}%
    \caption{#3}%
    \label{#4}%
    \end{center}%
  \end{figure}%
}
% Para las tablas
\newcommand{\otoprule}{\midrule [\heavyrulewidth]}
%
% Nuevo comando para tablas pequeñas (menos de una página).
\newcommand{\tablaSmall}[5]{%
 \begin{table}
  \begin{center}
   \rowcolors {2}{gray!35}{}
   \begin{tabular}{#2}
    \toprule
    #4
    \otoprule
    #5
    \bottomrule
   \end{tabular}
   \caption{#1}
   \label{tabla:#3}
  \end{center}
 \end{table}
}

%
% Nuevo comando para tablas pequeñas (menos de una página).
\newcommand{\tablaSmallSinColores}[5]{%
 \begin{table}[H]
  \begin{center}
   \begin{tabular}{#2}
    \toprule
    #4
    \otoprule
    #5
    \bottomrule
   \end{tabular}
   \caption{#1}
   \label{tabla:#3}
  \end{center}
 \end{table}
}

\newcommand{\tablaApaisadaSmall}[5]{%
\begin{landscape}
  \begin{table}
   \begin{center}
    \rowcolors {2}{gray!35}{}
    \begin{tabular}{#2}
     \toprule
     #4
     \otoprule
     #5
     \bottomrule
    \end{tabular}
    \caption{#1}
    \label{tabla:#3}
   \end{center}
  \end{table}
\end{landscape}
}

%
% Nuevo comando para tablas grandes con cabecera y filas alternas coloreadas en gris.
\newcommand{\tabla}[6]{%
  \begin{center}
    \tablefirsthead{
      \toprule
      #5
      \otoprule
    }
    \tablehead{
      \multicolumn{#3}{l}{\small\sl continúa desde la página anterior}\\
      \toprule
      #5
      \otoprule
    }
    \tabletail{
      \hline
      \multicolumn{#3}{r}{\small\sl continúa en la página siguiente}\\
    }
    \tablelasttail{
      \hline
    }
    \bottomcaption{#1}
    \rowcolors {2}{gray!35}{}
    \begin{xtabular}{#2}
      #6
      \bottomrule
    \end{xtabular}
    \label{tabla:#4}
  \end{center}
}

%
% Nuevo comando para tablas grandes con cabecera.
\newcommand{\tablaSinColores}[6]{%
  \begin{center}
    \tablefirsthead{
      \toprule
      #5
      \otoprule
    }
    \tablehead{
      \multicolumn{#3}{l}{\small\sl continúa desde la página anterior}\\
      \toprule
      #5
      \otoprule
    }
    \tabletail{
      \hline
      \multicolumn{#3}{r}{\small\sl continúa en la página siguiente}\\
    }
    \tablelasttail{
      \hline
    }
    \bottomcaption{#1}
    \begin{xtabular}{#2}
      #6
      \bottomrule
    \end{xtabular}
    \label{tabla:#4}
  \end{center}
}

%
% Nuevo comando para tablas grandes sin cabecera.
\newcommand{\tablaSinCabecera}[5]{%
  \begin{center}
    \tablefirsthead{
      \toprule
    }
    \tablehead{
      \multicolumn{#3}{l}{\small\sl continúa desde la página anterior}\\
      \hline
    }
    \tabletail{
      \hline
      \multicolumn{#3}{r}{\small\sl continúa en la página siguiente}\\
    }
    \tablelasttail{
      \hline
    }
    \bottomcaption{#1}
  \begin{xtabular}{#2}
    #5
   \bottomrule
  \end{xtabular}
  \label{tabla:#4}
  \end{center}
}



\definecolor{cgoLight}{HTML}{EEEEEE}
\definecolor{cgoExtralight}{HTML}{FFFFFF}

%
% Nuevo comando para tablas grandes sin cabecera.
\newcommand{\tablaSinCabeceraConBandas}[5]{%
  \begin{center}
    \tablefirsthead{
      \toprule
    }
    \tablehead{
      \multicolumn{#3}{l}{\small\sl continúa desde la página anterior}\\
      \hline
    }
    \tabletail{
      \hline
      \multicolumn{#3}{r}{\small\sl continúa en la página siguiente}\\
    }
    \tablelasttail{
      \hline
    }
    \bottomcaption{#1}
    \rowcolors[]{1}{cgoExtralight}{cgoLight}

  \begin{xtabular}{#2}
    #5
   \bottomrule
  \end{xtabular}
  \label{tabla:#4}
  \end{center}
}


















\graphicspath{ {./img/} }

% Capítulos
\chapterstyle{bianchi}
\newcommand{\capitulo}[2]{
	\setcounter{chapter}{#1}
	\setcounter{section}{0}
	\chapter*{#2}
	\addcontentsline{toc}{chapter}{#2}
	\markboth{#2}{#2}
}

% Apéndices
\renewcommand{\appendixname}{Apéndice}
\renewcommand*\cftappendixname{\appendixname}

\newcommand{\apendice}[1]{
	%\renewcommand{\thechapter}{A}
	\chapter{#1}
}

\renewcommand*\cftappendixname{\appendixname\ }

% Formato de portada
\makeatletter
\usepackage{xcolor}
\newcommand{\tutor}[1]{\def\@tutor{#1}}
\newcommand{\course}[1]{\def\@course{#1}}
\definecolor{cpardoBox}{HTML}{E6E6FF}
\def\maketitle{
  \null
  \thispagestyle{empty}
  % Cabecera ----------------
\noindent\includegraphics[width=\textwidth]{cabecera}\vspace{1cm}%
  \vfill
  % Título proyecto y escudo informática ----------------
  \colorbox{cpardoBox}{%
    \begin{minipage}{.8\textwidth}
      \vspace{.5cm}\Large
      \begin{center}
      \textbf{TFG del Grado en Ingeniería Informática}\vspace{.6cm}\\
      \textbf{\LARGE\@title{}}
      \end{center}
      \vspace{.2cm}
    \end{minipage}

  }%
  \hfill\begin{minipage}{.20\textwidth}
    \includegraphics[width=\textwidth]{escudoInfor}
  \end{minipage}
  \vfill
  
  
  % Datos de alumno, curso y tutores ------------------
  \begin{center}%
  {%
  	\imagenPequena{portada/logo}
    \noindent\LARGE
    Presentado por \@author{}\\ 
    en Universidad de Burgos --- \@date{}\\
    Tutores: \@tutor{}\\
  }%
  \end{center}%
  \null
  \cleardoublepage
  }
\makeatother

\newcommand{\nombre}{Félix Movilla Alonso} %%% cambio de comando
\newcommand{\nombreTutor}{Pedro Renedo Fernández y Antonio Jesús Canepa Oneto}

% Datos de portada
\title{ARBUBU}
\author{\nombre}
\tutor{\nombreTutor}
\date{\today}

\begin{document}

\maketitle

%%%%%%%%%%%%%%%%%%%%%%%%%%%%%%%%%%%%%%%%%%%%%%%%%%%%%%%%%%%%%%%%%%%%%%%%%%%%%%%%%%%%%%%%


\noindent\includegraphics[width=\textwidth]{cabecera}\vspace{1cm}

\noindent D. Pedro Renedo Fernández, profesor del departamento de , área de Lenguajes y Sistemas Informáticos.

\noindent Expone:

\noindent Que el alumno D. \nombre, con DNI 71294724Z, ha realizado el Trabajo final de Grado en Ingeniería Informática titulado ARBUBU. 

\noindent Y que dicho trabajo ha sido realizado por el alumno bajo la dirección del que suscribe, en virtud de lo cual se autoriza su presentación y defensa.

\begin{center} %\large
En Burgos, {\large \today}
\end{center}

\vfill\vfill\vfill

% Author and supervisor
\begin{minipage}{0.45\textwidth}
\begin{flushleft} %\large
Vº. Bº. del Tutor:\\[2cm]
D. Pedro Renedo Fernández
\end{flushleft}
\end{minipage}
\hfill
\begin{minipage}{0.45\textwidth}
\begin{flushleft} %\large
Vº. Bº. del co-tutor:\\[2cm]
D. Antonio Jesús Canepa Oneto
\end{flushleft}
\end{minipage}
\hfill

\vfill


\frontmatter

% Abstract en castellano
\renewcommand*\abstractname{Resumen}
\begin{abstract}
A medida que va pasando el tiempo vemos que es más importante nuestra concienciación con el medio ambiente. Esto es lo que nuestros amigos de UBUVerde tratan de inculcarnos.

Con este proyecto hemos tratado de que todos los alumnos de la Universidad de Burgos tengan acceso a la localización de los árboles singulares de Burgos, así como una descripción de los aspectos más importantes de cada uno de ellos, mediante un diseño web.

Para realizar este proyecto hemos trabajado con Python, que es un lenguaje de programación perfecto para lo que hemos realizado.
Python tiene una gran comunidad a sus espaldas y está en constante evolución.

Además hemos necesitado de un framework de desarrollo web, el elegido ha sido Django, ya que está enteramente escrito en Python y es de código abierto.

Por último, hemos necesitado una base de datos donde guardar las características de nuestros arboles, hemos elegido sqlite3, que es la base de datos que viene ya incorporada con nuestro framework.


\end{abstract}

\renewcommand*\abstractname{Descriptores}
\begin{abstract}
UBUVerde, Python, Django, Sqlite3, diseño web \ldots
\end{abstract}

\clearpage

% Abstract en inglés
\renewcommand*\abstractname{Abstract}
\begin{abstract}
As time goes by we can see the importance about the environment and how we can help it. This is that UBUVerde's people try to teach us.

With this project we want that all the UBU's students have access to singular trees location in Burgos and the descriptions about their most important details through a web design system.

To work to this project we have used Python, because it uses the perfectly programming language for what we want to do. 
Python has many supports in the community and experience. It is always in constant evolution.

Futhermore we have needed a framework of web development. The chosen one has been Django because it is completely written by Python, and it has a code opened.

Finally, we have needed a database where we can save the tree's details. We have chosen Sqlite3, because it's the database we have in our framework.
\end{abstract}

\renewcommand*\abstractname{Keywords}
\begin{abstract}
UBUVerde, Python, Django, Sqlite3, web design  \ldots
\end{abstract}

\clearpage

% Indices
\tableofcontents

\clearpage

\listoffigures

\clearpage


\mainmatter
\capitulo{1}{Introducción}

\section{Descripción del contenido del trabajo}
Ante la creciente demanda de las personas por tener identificado cada cosa que le rodea, surgió la idea de Arbubu.

Arbubu trata de poner en conocimiento de las personas los diferentes árboles singulares que se encuentran en las zonas universitarias, en un futuro podría expandirse a toda la ciudad de Burgos o a otras ciudades o incluso a otro tipo de ámbitos, pero de ello hablaremos más adelante en Conclusiones Lineas de trabajo futuras
\ref{trabajosFuturos}. 

Con todo esto tratamos de que los alumnos universitarios y las personas de todos los ámbitos y edades tengan a un solo click la información de los árboles singulares que les rodean.

Para facilitar la búsqueda de dichos árboles hemos incorporado un mapa interactivo que nos muestra en tiempo real donde están situados cada uno de ellos, con una ventana de información de la especie y del propio individuo.

Se incorpora una interfaz intuitiva en la cual los usuarios podrán ir buscando paso a paso el árbol que desean buscar, pueden empezar la búsqueda por el tipo de familia e ir bajando hasta el individuo que deseen o bien empezar por el género, especie o directamente por la ubicación en las zonas universitarias y seleccionar el árbol deseado.
\newpage


\section{Estructura de la memoria}
La memoria se ha estructurado siguiendo los siguientes apartados:
\begin{itemize}
	\item \textbf{Introducción}: Descripción del contenido del trabajo.
	\item \textbf{Objetivos del Proyecto}: Explicación de los objetivos generales y técnicos del proyecto.
	\item \textbf{Conceptos teóricos}: Explicación de los principales conceptos teóricos.
	\item \textbf{Técnicas y herramientas}: Descripción breve y concisa de las técnicas y herramientas utilizadas en el proyecto.
	\item \textbf{Aspectos relevantes del desarrollo del proyecto}: Explicación y desarrollo de los aspectos más relevantes del proyecto.
	\item \textbf{Trabajos relacionados}: Descripción de trabajos que tengan cierto parecido al nuestro.
	\item \textbf{Conclusiones lineas de trabajo futuras}: Descripción de las posibles líneas de trabajo futuras y conclusiones del proyecto.
\end{itemize}

\section{Estructura de los anexos}
\begin{itemize}
	\item \textbf{Plan de proyecto}: Planificación temporal y estudio de la viabilidad económica y legal del proyecto.
	\item \textbf{Requisitos}: Especificación de los requisitos que se establecen al principio del proyecto.
	\item \textbf{Diseño}: Muestra la información relacionada con el diseño de la interfaz además del diseño de clases.
	\item \textbf{Manual del programador}: Recoge la instalación de herramientas, la compilación, ejecución del proyecto y pruebas.
	\item \textbf{Manual del usuario}: Guía de usuario con instrucciones que puedan facilitar el correcto manejo de la aplicación.
\end{itemize}

\section{Contenido del Cd}
\begin{itemize}
	\item \textbf{Memoria}: Contenido de la memoria en formato pdf.
	\item \textbf{Anexos}: Contenido de los anexos en formato pdf.
	\item \textbf{Vídeo explicativo}: Vídeo explicando el funcionamiento básico de la aplicación web.
	\item \textbf{Código}: Versión del código más reciente de la aplicación web.  	
\end{itemize}
\capitulo{2}{Objetivos del proyecto}

El principal objetivo del proyecto es realizar un diseño web, en el cual se puedan ver los árboles singulares de las zonas universitarias de Burgos, con sus principales características.

A través de un mapa podremos ver donde están ubicados los árboles. 
\section{Objetivos Generales}
\begin{itemize}
	\item Observar en un mapa los árboles singulares de las zonas universitarias de Burgos y ver sus características.
	\item Filtrar a través de la familia, género, especie y ubicación los distintos árboles.
	\item Loguearnos como usuario y ser capaces de introducir y descargar datos de los árboles.
	\item Realizar una primera toma de contacto con la búsqueda de árboles, que en una futura mejora no solo busquemos arboles, es decir, que seamos capaces de buscar monumentos, lugares importantes \ldots
	\item Poder compartir y dar me gusta a la página de facebook de UbuVerde e interactuar con ellos a través de twitter. 
	\item Estar en constante contacto con UbuVerde a través de 'Acerca de Nosotros' que es un enlace a la página de UbuVerde.
	
	\imagen{objetivos/portada}{Portada de la página web}
\end{itemize}
\newpage

\section{Objetivos Técnicos}
\begin{itemize}
	\item Crear la Base de Datos para guardar la información de las familias, géneros, especies e individuos con una estructura adecuada de tablas y campos.
	\item Ser capaz de introducir nuevos individuos en la base de datos Sqlite3.
	\item Ser capaz de descargar información de la base de datos Sqlite3.
	\item Plasmar en el mapa esos datos introducidos en la base de datos. 
	\item Desarrollar el diseño web con Django quedando plasmado con una interfaz gráfica amena y agradable para el usuario.
	\item Guardar en un repositorio de GitHub los cambios que hemos ido realizando.
	 
\end{itemize}


\capitulo{3}{Conceptos teóricos}

En esta sección vamos a hablar de la forma en el cual vamos a clasificar nuestros modelos de datos, así como los conceptos teóricos relacionados con el proyecto.

\section{Estructura de datos}

Todos los seres vivos podemos clasificarnos según unas categorías taxonómicas \cite{CategoriaTaxonomica}, pero en este caso nos vamos a centrar en lo va el proyecto, que no es otro que los árboles.

Hemos elegido esta estructura de los datos porque con ellos somos capaces de englobar todos los aspectos que más se adecuan a nuestro proyecto.

No hemos decidido introducir más modelos de entidades ya que los comunes son los especificados y todo lo que fuera añadir alguna entidad más no nos permitiría desarrollar el proyecto de la mejor manera posible.

En nuestro proyecto vamos a clasificar los árboles de esta manera:
\begin{itemize}
	\item Familia: Es la agrupación de árboles que se encuentran en un orden, por características comunes entre ellos.
	\item Género: De las familias provienen los géneros, conjuntos de especies relacionadas entre sí por características comunes.
	\item Especie: Es un grupo de individuos con las mismas características.
	\item Individuo: Son cada uno de los árboles, con unas características particulares.
\end{itemize}

\section{Leaflet}

Leaflet \cite{leaflet} es una biblioteca JavaScript de código abierto utilizada para plasmar en nuestra página web mapas interactivos, los cuales son compatibles con dispositivos móviles.

Se caracteriza por la sencillez, simplicidad, rendimiento y usabilidad, lo que hace que sea una herramienta perfecta para nuestro proyecto.

También es posible añadirle infinidad de plugins que lo hace todavía más completo.

Para verlo con más claridad incluyo una imagen \ref{fig:conceptosTeoricos/leaflet} en la cual podemos observar un mapa con los países por los que ha pasado o vivido el titular del blog Andy Maloney.

\insertarimagen{conceptosTeoricos/leaflet}{Ejemplo mapa interactivo}{EjemploLeaflet}


\capitulo{4}{Técnicas y herramientas}
Esta sección vamos a hablar de las técnicas metodológicas  y de las herramientas de desarrollo seguidas durante el proyecto.

\section{Técnicas Metodológicas}
No hemos seguido una metodología pura, es decir, no me he basado simplemente en una sola metodología si no que he ido eligiendo aspectos de varias de ellas.

En primer lugar, tomamos aspectos de la metodología en cascada\cite{ModeloenCascada}, ya que hemos partido de unos requisitos iniciales, que posteriormente hemos ido adaptando según los cambios y necesidades del cliente.

Más adelante una vez que teníamos claros los requisitos iniciales, nos hemos centrado más en la metodología scrum\cite{MetodologiaScrum}. Decimos que utilizamos esta metodologia porque a lo largo del desarrollo del proyecto hemos tenido reuniones semanales con los tutores, además de una reunión inicial con el encargado de Ubu Verde.

En todo momento hemos ido mezclando ambas técnicas ya que a las reuniones semanales con los tutores se fueron añadiendo a continuación el diseño e implementación de la aplicación.

Una vez que el diseño e implementación estaban realizados fuimos añadiendo las distintas pruebas para comprobar que no dejábamos cabos sueltos.

A medida que íbamos realizando las distintas partes del proyecto se iban subiendo al repositorio de Github.

\section{Herramientas de Desarrollo}

\subsection{Lenguaje de Programación}
Lo primero antes de elegir las herramientas que vamos a utilizar es elegir el lenguaje de programación, tenemos infinidad de lenguajes para desarrollar nuestro proyecto, pero los más viables para ello creí que eran Python\cite{Python} y Php\cite{Php}.


Me decanté por Python ya que es un lenguaje que me entra más por la vista y es más intuitivo, es un lenguaje más nuevo que Php por lo que en un futuro cuando trabaje habrá menos gente que conozca este lenguaje, es decir, menos competencia, existe una gran comunidad con gran cantidad de tutoriales \ldots

\subsection{Framework}

A la hora de decidir entre que framework elegir, me plantee dos posibles opciones que fueron Flask\cite{Flask} y Django\cite{Django}, finalmente me decidí por Django porque me pareció un framework más avanzado, te facilita mucho su desarrollo, ya que gran parte del código viene implementada y no es necesario programarlo, es el más utilizado por lo tanto es el que más comunidad tendrá a sus espaldas en caso de fallo o duda, es seguro ya que implementa medidas de seguridad por defecto y evita fallos como el SQL Injection, incluye una interfaz para acceder a la base de datos \ldots


\subsection{Base de Datos}

Existen 4 posibles opciones de bases de datos para elegir con Django:
	\begin{itemize}
	\item PostgreSQL\cite{PostgreSQL} es un sistema de gestión de bases de datos relacional orientado a objetos y de código abierto. 
	\item SQLite 3\cite{SQLite3} es un sistema de gestión de bases de datos relacional, el conjunto de la base de datos (definiciones, tablas, índices, y los propios datos), son guardados como un solo fichero estándar en la máquina host. Permite bases de datos de hasta 2 Terabytes de tamaño, y también permite la inclusión de campos tipo BLOB.
	\item MySQL\cite{MySQL} es un sistema de gestión de bases de datos relacional desarrollado bajo licencia dual: Licencia pública general/Licencia comercial por Oracle Corporation y está considerada como la base de datos de código abierto más popular del mundo.
	\item Oracle\cite{Oracle} es un sistema de gestión de base de datos de tipo objeto-relacional, su dominio en el mercado de servidores empresariales había sido casi total hasta que recientemente tiene la competencia del Microsoft SQL Server y de la oferta de otros RDBMS con licencia libre como PostgreSQL, MySQL o Firebird.
	\end{itemize}

Finalmente me decanté por SQLite 3 ya que es la que viene implementada por defecto con Django, es la más sencilla de las 4 pero para el proyecto que estamos desarrollando es más que suficiente.

\subsection{IDE}

Pycharm, atom

\subsection{Servidor Web}


\subsection{Repositorio}
github

\subsection{Documentación}

Latex, Word, OpenOffice

\subsection{Diagramas}
StarUml

\capitulo{5}{Aspectos relevantes del desarrollo del proyecto}


En este apartado vamos a describir y comentar los pasos realizados para el correcto funcionamiento de nuestro proyecto.

Mencionaremos también la formación realizada y los distintos documentos, vídeos y demás materiales utilizados.


\section{Pasos a realizar}

En primer lugar debemos instalar Python, que es el lenguaje de programación que vamos a utilizar, en nuestro caso vamos a utilizar Python 3.7.3 Ver figura \ref{fig:aspectosRelevantes/versionPython} aunque cualquier versión de Python superior a 3.0 nos valdría.

\imagen{aspectosRelevantes/versionPython}{Versión de Python utilizada}

\section{Formación}

Como en todo proyecto en el cual nos embarcamos es necesario una formación extra que afiance los conocimientos adquiridos durante nuestra vida y más concretamente en nuestra estancia en el Grado.

Vamos a explicar paso a paso los distintos documentos de texto y gráficos consultados para el correcto aprendizaje de todas las herramientas utilizadas:

\begin{itemize}
	\item \textbf{Curso de Django de pildorasinformaticas} \cite{djangoPildoras}: Son una serie de vídeos en youtube donde un famoso youtuber y profesor explica desde el principio como crear un proyecto con Django e ir avanzando hasta cosas más complejas.
	\item \textbf{Tutorial Leaflet} \cite{tutorialLeaflet}: Son una serie de lecciones desde la configuración inicial hasta la realización de un mapa dinámico. 
	\item \textbf{Tutorial Python} \cite{tutorialPython}: Son una serie de lecciones desde como instalar Python a como instalar un módulo. 
	\item \textbf{Curso Django con Udemy} \cite{udemy}: Es un curso super básico de Django para aprender a programar páginas web.
	\item \textbf{Curso no convencional de LaTeX} \cite{cursoLatex}: Es un repositorio en GitHub que abarca numerosos contenidos de LaTeX. 
	\item \textbf{Curso CSS Avanzado} \cite{cursoCss}: Son una serie de vídeos en youtube donde un famoso youtuber y profesor explica desde el principio como crear un CSS e ir avanzando hasta cosas más complejas.
	\item \textbf{Curso HTML5} \cite{cursoHTML}: Son una serie de vídeos en youtube donde un famoso youtuber y profesor explica desde el principio como utilizar HTML5 e ir avanzando hasta cosas más complejas.
\end{itemize}
\capitulo{6}{Trabajos relacionados}

A continuación explicaremos las distintas páginas o documentos observados que están relacionados con el proyecto: 

\section{ArbolApp}

ArbolApp \cite{ArbolApp} es una aplicación destinada para dar a conocer los distintos árboles y sus especies en la Península Ibérica e Islas Canarias, es un proyecto elaborado por el Real Jardín Botánico y el Área de Cultura Científica del CSIC.
Ver figura \ref{fig:trabajosRelacionados/arbolapp}
\imagen{trabajosRelacionados/arbolapp}{Foto Portada Página Web de ArbolApp}

\section{Observation}

Observation.org \cite{Observation} es una página web muy amplia que está destinada a ver las observaciones que cada usuario quiera publicar de aves, mamíferos, plantas \ldots, un apartado para observar en detalle cada individuo observado por algún usuario, un apartado de especies observadas con distintas fotos e información de ellas y un apartado destinado a ver en cada comunidad autónoma un mapa con los distintos individuos observados.
Ver figura \ref{fig:trabajosRelacionados/observado}
\imagen{trabajosRelacionados/observado}{Foto Portada Página Web de Observado}

\section{Árboles Monumentales}

Árboles Monumentales \cite{arbolesMonumentales} es una página web donde obtener información de infinidad de árboles que están registrados en su base de datos, se puede ver la localización de dichos árboles con fotos y una pequeña explicación de sus características.

Está en continuo movimiento la página ya que los propios usuarios pueden registrarse e introducir nuevos datos de árboles que no estuvieran en la página.
Ver figura \ref{fig:trabajosRelacionados/arbolesMonumentales}
\imagen{trabajosRelacionados/arbolesMonumentales}{Foto Portada Página Web de Árboles Monumentales}

\section{Árboles Singulares}

Árboles Singulares de la ciudad de Burgos \cite{arbolesSingulares} es un documento del ayuntamiento de Burgos, el cual recoge los árboles que por sus características se podrían catalogar como singulares. Ver figura \ref{fig:trabajosRelacionados/arbolesSingulares}
\imagen{trabajosRelacionados/arbolesSingulares}{Foto Portada Documento de Árboles Singulares}
\capitulo{7}{Conclusiones y Líneas de trabajo futuras}

\section{Conclusiones} 

Tras haber acabado con el proyecto se puede decir que ha cumplido la gran mayoría de objetivos y requisitos que nos habían propuesto.

Como consecuencia del desarrollo del proyecto he aprendido a utilizar herramientas como Django, SQLite 3 y he perfeccionado mis conocimientos en Python, Css y HTML.

Por otra parte, he aprendido gran cantidad de aspectos relacionados con los árboles y me ha parecido un tema muy ameno y con el que no me importaría seguir trabajando en un futuro.

Para finalizar puedo afirmar que estoy muy contento con el trabajo realizado, siendo una experiencia muy positiva académica y personalmente ya que he trabajado con unos tutores que me han ayudado en todo lo que han podido.

\section{Líneas de trabajo futuras} \label{trabajosFuturos}

Este proyecto puede ser tan grande como el diseñador y programador quiera, es decir, no tiene por qué limitarse a ser un proyecto basado únicamente en árboles, puede basarse en animales avistados, monumentos de gran patrimonio cultural, bares históricos \ldots 

Lógicamente únicamente con los modelos creados no podríamos abarcar todos estos temas anteriores, habría que crear nuevos modelos de datos y reutilizar los que si nos podrían valer de los ya realizados.

A medida que el proyecto fuese creciendo veríamos que la base de datos elegida (SQLite 3) se va quedando pequeña, ya que no admite una gran cantidad de datos y sería necesario cambiar a otra como Oracle o MySql, que no supondría ningún inconveniente, ya que es cambiar unas pequeñas líneas de código.  

Otra posible mejora es la internacionalización de la página web, es decir, que pueda cambiarse de un idioma a otro según la procedencia de la persona que está buscando la información.

Además este proyecto se puede mejorar realizando una aplicación móvil, que se realizará en futuros proyectos.

 



\bibliographystyle{plain}
\bibliography{bibliografia}

\end{document}