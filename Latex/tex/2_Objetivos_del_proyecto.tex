\capitulo{2}{Objetivos del proyecto}

El principal objetivo del proyecto es realizar un diseño web, en el cual se puedan ver los árboles singulares de las zonas universitarias de Burgos, con sus principales características.

A través de un mapa podremos ver donde están ubicados los árboles. 
\section{Objetivos Generales}
\begin{itemize}
	\item Observar en un mapa los árboles singulares de las zonas universitarias de Burgos y ver sus características.
	\item Filtrar a través de la familia, nombre científico y nombre común de la especie, autóctona y motivo singular donde se sitúan los árboles buscados y sus características.
	\item Loguearnos como usuario y ser capaces de importar y descargar datos de los árboles.
	\item Realizar una primera toma de contacto con la búsqueda de árboles, que en una futura mejora no solo busquemos arboles, es decir, que seamos capaces de buscar monumentos, lugares importantes \ldots
	\item Poder compartir y dar me gusta a la página de facebook de UbuVerde e interactuar con ellos a través de twitter. 
	%Añadir imagen de la pagina cuando este terminada
\end{itemize}
\newpage

\section{Objetivos Técnicos}
\begin{itemize}
	\item Ser capaz de introducir datos en la base de datos Sqlite3.
	\item Ser capaz de descargar datos de la base de datos Sqlite3.
	\item Plasmar en el mapa esos datos introducidos en la base de datos. 
	\item Programar en Python y html el diseño web que va a tener nuestro proyecto.
	\item Guardar en un repositorio de GitHub los cambios que hemos ido realizando.
	\item Utilizar el framework Django para realizar correctamente nuestro diseño web.
	 
\end{itemize}