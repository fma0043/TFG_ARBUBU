\capitulo{6}{Trabajos relacionados}

A continuación explicaremos las distintas páginas o documentos observados que están relacionados con el proyecto: 

\section{ArbolApp}

ArbolApp \cite{ArbolApp} es una aplicación destinada para dar a conocer los distintos árboles y sus especies en la Península Ibérica e Islas Canarias, es un proyecto elaborado por el Real Jardín Botánico y el Área de Cultura Científica del CSIC.
Ver figura \ref{fig:trabajosRelacionados/arbolapp}
\imagen{trabajosRelacionados/arbolapp}{Foto Portada Página Web de ArbolApp}

\section{Observation}

Observation.org \cite{Observation} es una página web muy amplia que está destinada a ver las observaciones que cada usuario quiera publicar de aves, mamíferos, plantas \ldots, un apartado para observar en detalle cada individuo observado por algún usuario, un apartado de especies observadas con distintas fotos e información de ellas y un apartado destinado a ver en cada comunidad autónoma un mapa con los distintos individuos observados.
Ver figura \ref{fig:trabajosRelacionados/observado}
\imagen{trabajosRelacionados/observado}{Foto Portada Página Web de Observado}

\section{Árboles Monumentales}

Árboles Monumentales \cite{arbolesMonumentales} es una página web donde obtener información de infinidad de árboles que están registrados en su base de datos, se puede ver la localización de dichos árboles con fotos y una pequeña explicación de sus características.

Está en continuo movimiento la página ya que los propios usuarios pueden registrarse e introducir nuevos datos de árboles que no estuvieran en la página.
Ver figura \ref{fig:trabajosRelacionados/arbolesMonumentales}
\imagen{trabajosRelacionados/arbolesMonumentales}{Foto Portada Página Web de Árboles Monumentales}

\section{Árboles Singulares}

Árboles Singulares de la ciudad de Burgos \cite{arbolesSingulares} es un documento del ayuntamiento de Burgos, el cual recoge los árboles que por sus características se podrían catalogar como singulares. Ver figura \ref{fig:trabajosRelacionados/arbolesSingulares}
\imagen{trabajosRelacionados/arbolesSingulares}{Foto Portada Documento de Árboles Singulares}