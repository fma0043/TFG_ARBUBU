\capitulo{5}{Aspectos relevantes del desarrollo del proyecto}


En este apartado vamos a describir y comentar los pasos realizados para el correcto funcionamiento de nuestro proyecto.

Mencionaremos también la formación realizada y los distintos documentos, vídeos y demás materiales utilizados.


\section{Pasos a realizar}

En primer lugar debemos instalar Python, que es el lenguaje de programación que vamos a utilizar, en nuestro caso vamos a utilizar Python 3.7.3 Ver figura \ref{fig:aspectosRelevantes/versionPython} aunque cualquier versión de Python superior a 3.0 nos valdría.

\imagen{aspectosRelevantes/versionPython}{Versión de Python utilizada}

\section{Formación}

Como en todo proyecto en el cual nos embarcamos es necesario una formación extra que afiance los conocimientos adquiridos durante nuestra vida y más concretamente en nuestra estancia en el Grado.

Vamos a explicar paso a paso los distintos documentos de texto y gráficos consultados para el correcto aprendizaje de todas las herramientas utilizadas:

\begin{itemize}
	\item \textbf{Curso de Django de pildorasinformaticas} \cite{djangoPildoras}: Son una serie de vídeos en youtube donde un famoso youtuber y profesor explica desde el principio como crear un proyecto con Django e ir avanzando hasta cosas más complejas.
	\item \textbf{Tutorial Leaflet} \cite{tutorialLeaflet}: Son una serie de lecciones desde la configuración inicial hasta la realización de un mapa dinámico. 
	\item \textbf{Tutorial Python} \cite{tutorialPython}: Son una serie de lecciones desde como instalar Python a como instalar un módulo. 
	\item \textbf{Curso Django con Udemy} \cite{udemy}: Es un curso super básico de Django para aprender a programar páginas web.
	\item \textbf{Curso no convencional de LaTeX} \cite{cursoLatex}: Es un repositorio en GitHub que abarca numerosos contenidos de LaTeX. 
	\item \textbf{Curso CSS Avanzado} \cite{cursoCss}: Son una serie de vídeos en youtube donde un famoso youtuber y profesor explica desde el principio como crear un CSS e ir avanzando hasta cosas más complejas.
	\item \textbf{Curso HTML5} \cite{cursoHTML}: Son una serie de vídeos en youtube donde un famoso youtuber y profesor explica desde el principio como utilizar HTML5 e ir avanzando hasta cosas más complejas.
\end{itemize}