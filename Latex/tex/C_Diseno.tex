\apendice{Especificación de diseño}

\section{Introducción}

En este apartado explicaremos el diseño que ha dado lugar a la aplicación, para ello el anexo está dividido en tres secciones: el diseño de datos, el diseño procedimental y el diseño arquitectónico.

\section{Diseño de datos}

Para el almacenaje de los datos de la aplicación he utilizado una base de datos sqlite3, la cual está compuesta por seis tablas o modelos.

\imagen{diseño/familias}{Tabla Familia}

\begin{itemize}
	\item \textbf{FAMILIA}: En este modelo se va a guardar la información relacionada con las familias de árboles existentes en las universidades de Burgos, está compuesta por dos campos que son:
	\begin{itemize}
		\item \textbf{idFamilia}: Es la clave primaria de la tabla y contiene un identificador único para cada familia.
		\item \textbf{nombreFamilia}: Contiene el nombre de la familia, es un campo de tipo texto, tiene que ser único, no se puede dejar en blanco y con un tamaño máximo de 30 caracteres. 
	\end{itemize}
\end{itemize}

\imagen{diseño/generos}{Tabla Genero}

\begin{itemize}
	\item \textbf{GENERO}: En este modelo se va a guardar la información relacionada con los géneros de árboles existentes en las universidades de Burgos, está compuesta por tres campos que son:
	\begin{itemize}
		\item \textbf{idGenero}: Es la clave primaria de la tabla y contiene un identificador único para cada género.
		\item \textbf{nombreFamilia}: Contiene el nombre del género, es un campo de tipo texto, tiene que ser único, no se puede dejar en blanco y con un tamaño máximo de 30 caracteres. 
		\item \textbf{familia}: Contiene el nombre de la familia, es una FK.
	\end{itemize}
\end{itemize}

\imagen{diseño/especies}{Tabla Especie}

\begin{itemize}
	\item \textbf{ESPECIE}: En este modelo se va a guardar la información relacionada con las especies de árboles existentes en las universidades de Burgos, está compuesta por siete campos que son:
	\begin{itemize}
		\item \textbf{idEspecie}: Es la clave primaria de la tabla y contiene un identificador único para cada especie.
		\item \textbf{nombreCientificoEspecie}: Contiene el nombre científico de la especie, es un campo de tipo texto, no se puede dejar en blanco y con un tamaño máximo de 50 caracteres.
		\item \textbf{nombreComunEspecie}: Contiene el nombre común de la especie, es un campo de tipo texto, no se puede dejar en blanco y con un tamaño máximo de 50 caracteres.
		\item \textbf{genero}: Contiene el nombre del género, es una FK.
		\item \textbf{autoctona}: Contiene si la especie es autóctona, es un campo de tipo booleano y no se puede dejar en blanco.
		\item \textbf{descripcion}: Contiene una pequeña descripción de la especie, es un campo de tipo texto y no se puede dejar en blanco.
		\item \textbf{ecologia}: Contiene la ecología de la especie, es un campo de tipo texto y no se puede dejar en blanco.
	\end{itemize}
\end{itemize}

\imagen{diseño/individuos}{Tabla Individuos}

\begin{itemize}
	\item \textbf{INDIVIDUO}: En este modelo se va a guardar la información relacionada con cada uno de árboles existentes en las universidades de Burgos, está compuesta por catorce campos que son:
	\begin{itemize}
		\item \textbf{idIndividuo}: Es la clave primaria de la tabla y contiene un identificador único para cada individuo.
		\item \textbf{nombreComun}: Contiene el nombre común del árbol, es un campo de tipo texto, no se puede dejar en blanco y con un tamaño máximo de 30 caracteres. 
		\item \textbf{especie}: Contiene el nombre de la especie, es una FK.
		\item \textbf{motivoSingular}: Contiene un motivo de por qué el árbol es singular, es un campo de tipo texto, no se puede dejar en blanco y con un tamaño máximo de 50 caracteres. 
		\item \textbf{explicacionMotivoSingular}: Contiene una explicación más detallada de porque el árbol es singular, es un campo de tipo texto y no se puede dejar en blanco.
		\item \textbf{latitud}: Contiene la latitud del árbol, es un campo de tipo decimal y no se puede dejar en blanco.
		\item \textbf{longitud}: Contiene la longitud del árbol, es un campo de tipo decimal y no se puede dejar en blanco.
		\item \textbf{fotoArbol}: Contiene la foto del árbol, es un campo de tipo imagen y si se puede dejar en blanco.
		\item \textbf{fotoHojas}: Contiene la foto de las hojas del árbol, es un campo de tipo imagen y si se puede dejar en blanco.
		\item \textbf{fotoTronco}: Contiene la foto del tronco del árbol, es un campo de tipo imagen y si se puede dejar en blanco.
		\item \textbf{fotoFrutos}: Contiene la foto de los frutos del árbol, es un campo de tipo imagen y si se puede dejar en blanco.
		\item \textbf{ubicacion}: Contiene la ubicación del árbol, es decir, la zona universitaria donde está ubicado, es un campo de tipo texto, no se puede dejar en blanco y con un tamaño máximo de 50 caracteres.
		\item \textbf{altura}: Contiene la altura del árbol, es un campo de tipo decimal, si se puede dejar en blanco y con un tamaño máximo de 19 caracteres.
		\item \textbf{perimetro}: Contiene el perímetro del árbol, es un campo de tipo decimal, si se puede dejar en blanco y con un tamaño máximo de 19 caracteres.  
	\end{itemize}
\end{itemize}

\imagen{diseño/usuarios}{Tabla Usuario}

\begin{itemize}
	\item \textbf{USUARIO}: En este modelo se va a guardar la información relacionada con los usuarios, está compuesta por dos campos que son:
		\begin{itemize}
		\item \textbf{idUsuario}: Es la clave primaria de la tabla y contiene un identificador único para cada usuario.
		\item \textbf{usuario}: Contiene el nombre del usuario, es una FK. 
	\end{itemize}
\end{itemize}

\begin{itemize}
	\item \textbf{USER}: Este modelo está diseñado por el framework y guarda información relacionada con los usuarios, está compuesta por doce campos que son:
	\begin{itemize}
		\item \textbf{username}: Contiene el nombre del usuario, es un campo de tipo texto, tiene que ser único, no se puede dejar en blanco y con un tamaño máximo de 150 caracteres.
		\item \textbf{first name}: Contiene el nombre del usuario, es un campo de tipo texto, se puede dejar en blanco y con un tamaño máximo de 30 caracteres.
		\item \textbf{last name}: Contiene el apellido del usuario, es un campo de tipo texto, se puede dejar en blanco y con un tamaño máximo de 150 caracteres.  
		\item \textbf{email}: Contiene el email del usuario, es un campo de tipo email y se puede dejar en blanco.
		\item \textbf{password}: Contiene la contraseña del usuario, es un campo de tipo texto, es necesario y se encripta.
		\item \textbf{groups}: Contiene el grupo al cual pertenece el usuario. 
		\item \textbf{user permissions}: Contiene los permisos que tiene cada usuario.
		\item \textbf{is staff}: Nos indica si este usuario puede acceder al sitio de administración.
		\item \textbf{is active}: Nos indica si este usuario está activo.  
		\item \textbf{is superuser}: Nos indica si este usuario es super usuario.
		\item \textbf{last login}: Nos indica una fecha y hora del último inicio de sesión del usuario.
		\item \textbf{date joined}: Nos indica una fecha y hora de cuando se creó la cuenta del usuario. 
	\end{itemize}
	
\end{itemize}

\section{Diseño procedimental}

Para comprender y entender las funciones que va a desempeñar nuestra aplicación web he realizado un diagrama de navegabilidad \ref{fig:diseño/diagramaNavegabilidad}, el cual explicaré en detalle a continuación:

\begin{itemize}
	\item En primer lugar el usuario accede a la pantalla principal de la página.
	\item Una vez dentro tiene dos opciones, registrarse e iniciar sesión o navegar sin las ventajas de ser usuario.
	\item Si el usuario no inicia sesión podrá:
	\begin{itemize}
		\item Seguir en Twitter a @UbuVerde.
		\item Dar Me Gusta y compartir noticias de UbuVerde. 
		\item Visitar la página de UbuVerde (Acerca De Nosotros). 
		\item Ver el mapa con todos los árboles y seleccionar uno y ver sus características y fotos.
		\item Ver todas las familias:
		\begin{itemize}
			\item Elegir una o ver el mapa y seleccionar un árbol.
			\item Si seleccionas una familia, la ves y eliges un género.
			\item Una vez seleccionado el género, ves el género, la familia y eliges la especie.
			\item Ves las características de la especie y eliges el individuo.
			\item Ves las fotos, el mapa y las características del árbol seleccionado.
			\item Si seleccionas un árbol ves las características del árbol, las fotos y el mapa donde esta ubicado.
		\end{itemize} 
		\item Ver todos los géneros:
		\begin{itemize}
			\item Elegir uno o ver el mapa y seleccionar un árbol.
			\item Si seleccionas un género, ves la familia y el género seleccionado y eliges una especie.
			\item Ves las características de la especie y eliges el individuo.
			\item Ves las fotos, el mapa y las características del árbol seleccionado.
			\item Si seleccionas un árbol ves las características del árbol, las fotos y el mapa donde esta ubicado. 
		\end{itemize}
		\item Ver todas las especies:
		\begin{itemize}
			\item Elegir una o ver el mapa y seleccionar un árbol.
			\item Si seleccionas una especie, ves las características de la especie y eliges el individuo.
			\item Ves las fotos, el mapa y las características del árbol seleccionado.
			\item Si seleccionas un árbol ves las características del árbol, las fotos y el mapa donde esta ubicado.
		\end{itemize}
		\item Ver todos los individuos:
		\begin{itemize}
			\item Ves las 5 ubicaciones y seleccionas una.
			\item Ves el mapa con los árboles de esa ubicación y seleccionas uno.
			\item Ves las fotos, el mapa y las características del árbol seleccionado.
		\end{itemize}
	\end{itemize}
	\item Si el usuario inicia sesión podrá además de todas esas cosas anteriores:
	\begin{itemize}
		\item Descargar un pdf con las características de los árboles.
		\item Agregar nuevos árboles.
		\item Cerrar sesión.
	\end{itemize}
	
\end{itemize}

\imagen{diseño/diagramaNavegabilidad}{Diagrama de Navegabilidad}

\section{Diseño arquitectónico}

El proyecto como está realizado mediante el framework Django seguirá el patrón MVT\cite{MVT}.

Es un framework muy parecido al MVC, pero con ligeros matices:
\begin{itemize}
	\item \textbf{M}: Significa Model o Modelo, es la capa que tiene acceso a la base de datos. En ella esta contenida la información relacionada con los datos, es decir, cómo acceder a ellos, cómo se comportan, cómo se validan y las relaciones entre sí.
	\item \textbf{V}: Significa View o Vista, es la capa de la lógica de negocios. En ella esta contenida la lógica que accede al modelo y se comunica con el template o plantilla deseada. 
	\item \textbf{T}: significa Template o Plantilla, es la capa de presentación. En ella estará contenido las decisiones que tienen que ver con la presentación de los datos, en nuestro caso como se va a ver la página web.
\end{itemize}

Muchas veces se tiende a decir que las Vistas de Django se pueden asemejar al controlador y las Plantillas pueden ser las vistas en el MVC, pero esto no es cierto, en realidad la vista en MVC describe los datos que son presentados al usuario; no necesariamente el cómo se mostrarán, pero si cuáles datos son presentados.