\capitulo{4}{Técnicas y herramientas}
Esta sección vamos a hablar de las técnicas metodológicas  y de las herramientas de desarrollo seguidas durante el proyecto.

\section{Técnicas Metodológicas}
No hemos seguido una metodología pura, es decir, no me he basado simplemente en una sola metodología si no que he ido eligiendo aspectos de varias de ellas.

En primer lugar, tomamos aspectos de la metodología en cascada\cite{ModeloenCascada}, ya que hemos partido de unos requisitos iniciales, que posteriormente hemos ido adaptando según los cambios y necesidades del cliente.

Más adelante una vez que teníamos claros los requisitos iniciales, nos hemos centrado más en la metodología scrum\cite{MetodologiaScrum}. Decimos que utilizamos esta metodologia porque a lo largo del desarrollo del proyecto hemos tenido reuniones semanales con los tutores, además de una reunión inicial con el encargado de Ubu Verde.

En todo momento hemos ido mezclando ambas técnicas ya que a las reuniones semanales con los tutores se fueron añadiendo a continuación el diseño e implementación de la aplicación.

Una vez que el diseño e implementación estaban realizados fuimos añadiendo las distintas pruebas para comprobar que no dejábamos cabos sueltos.

A medida que íbamos realizando las distintas partes del proyecto se iban subiendo al repositorio de Github.

\section{Herramientas de Desarrollo}

\subsection{Lenguaje de Programación}
Lo primero antes de elegir las herramientas que vamos a utilizar es elegir el lenguaje de programación, tenemos infinidad de lenguajes para desarrollar nuestro proyecto, pero los más viables para ello creí que eran Python\cite{Python} y Php\cite{Php}.


Me decanté por Python ya que es un lenguaje que me entra más por la vista y es más intuitivo, es un lenguaje más nuevo que Php por lo que en un futuro cuando trabaje habrá menos gente que conozca este lenguaje, es decir, menos competencia, existe una gran comunidad con gran cantidad de tutoriales \ldots

\subsection{Framework}

A la hora de decidir entre que framework elegir, me plantee dos posibles opciones que fueron Flask\cite{Flask} y Django\cite{Django}, finalmente me decidí por Django porque me pareció un framework más avanzado, te facilita mucho su desarrollo, ya que gran parte del código viene implementada y no es necesario programarlo, es el más utilizado por lo tanto es el que más comunidad tendrá a sus espaldas en caso de fallo o duda, es seguro ya que implementa medidas de seguridad por defecto y evita fallos como el SQL Injection, incluye una interfaz para acceder a la base de datos \ldots


\subsection{Base de Datos}

Existen 4 posibles opciones de bases de datos para elegir con Django:
	\begin{itemize}
	\item PostgreSQL\cite{PostgreSQL} es un sistema de gestión de bases de datos relacional orientado a objetos y de código abierto. 
	\item SQLite 3\cite{SQLite3} es un sistema de gestión de bases de datos relacional, el conjunto de la base de datos (definiciones, tablas, índices, y los propios datos), son guardados como un solo fichero estándar en la máquina host. Permite bases de datos de hasta 2 Terabytes de tamaño, y también permite la inclusión de campos tipo BLOB.
	\item MySQL\cite{MySQL} es un sistema de gestión de bases de datos relacional desarrollado bajo licencia dual: Licencia pública general/Licencia comercial por Oracle Corporation y está considerada como la base de datos de código abierto más popular del mundo.
	\item Oracle\cite{Oracle} es un sistema de gestión de base de datos de tipo objeto-relacional, su dominio en el mercado de servidores empresariales había sido casi total hasta que recientemente tiene la competencia del Microsoft SQL Server y de la oferta de otros RDBMS con licencia libre como PostgreSQL, MySQL o Firebird.
	\end{itemize}

Finalmente me decanté por SQLite 3 ya que es la que viene implementada por defecto con Django, es la más sencilla de las 4 pero para el proyecto que estamos desarrollando es más que suficiente.

\subsection{IDE}

Pycharm, atom

\subsection{Servidor Web}


\subsection{Repositorio}
github

\subsection{Documentación}

Latex, Word, OpenOffice

\subsection{Diagramas}
StarUml
